% Lecture 6: 03 October 2012
\sektion{6}{Privacy, Security, and Identity Online}
Guest lecturer Arvind Narayanan

Security vs Privacy

Do we use encrpytion for security or privacy? Depends

Example: You're sitting in a coffee shop and due to a weakness in the encryption
a creepy guy two tables away is sniffing the wireless and reading your Gmail. Is
this a security or privacy breach? Class answer: both

Privacy of email but compromised by security bug in the software

\subsektion{Online tracking}
``On the Internet, nobody knows you're a dog.'' -- now a complete reversal from
this situation, and now sites are trying to collect as much information on you
as possible (ads, personalization, ...)

Behavioral targeting: on the webpages, cookies, javascript, webbugs (1 px gifs)
tracking you; the third party domain is getting your information.

Refer header: whenever a page embeds another page, the 3rd party page gets an
extra header (top level url of page you're looking at)

Ad Network as auction between users/publishers and advertisers
\begin{itemize}
\item places ads
\item collecting your data to get what kind of ad is suited to you, not just the
page you're on
\end{itemize}

Third Party Tracking

All the information is going into a database: NOT aggregate

Who has access? Tons of companies

64 indepdendent tracking mechanisms on average on top 50 sites.

Aliasing: visit hi5.com with subdomain ad.hi5.com but DNS redirects to
ad.yieldmanager.com. Browser tricked since this works even if you block 3rd
party cookies.

Also flash cookies and browser fingerprinting.

Fingerprinting since our browsers are unique enough: user agent string, plugins

See \href{https://panopticlick.eff.org/}{panopticlick}.

User-agent string along: 10 bits of entropy, 84\% of fingerprints unique, with
Flash or Java, 94\% unique.
\subsektion{Anonymity vs pseudonymity vs identity}
``Don't worry, it's all anonymous''

Truly anonymous shouldn't be able to track you under a pseudonym in a different
session. (How the Internet started)

Pseudononymous can tell when same person comes back but don't know real-life
identiy. (Internet post-cookies)

Identity can get back to real-life identity.

Ways for websites to get your identity:
\begin{enumerate}
    \item Third party is sometimes a first party: have first party relationship
        with social networking sites but they're also as widgets on other pages

        Example: Facebook's Like button -- even if you don't click it, Facebook
        knows you were on that page
    \item Leakage of identifiers:

        \begin{tt}
        GET http://ad.doubleclick.net/adj/...\\
        Referer: http://submit.SPORTS.com/...?email=\color{red}{jdoe@email.com}\\
        Cookie: id=\color{red}{35c192bcfe0000b1...}\\
        \end{tt}

        Identity has been compromised now and in the future
    \item Third party buys your identity: free iPod scam passing your email to
        first party site
    \item Security bugs:
        \begin{itemize}
        \item http://google.com/profiles/me redirects to
            http://google.com/profiles/randomwalker

            In firefox, can put the url in a script tag, JavaScript throws error
            which includes the url, giving randomwalker or other identity just
            from visiting this random page

            Mozilla's solution: only tell original, not redirected URL
        \item Google spreadsheets: people don't necessarily understand can be
            public

            Can specify all in URL in search to get public spreadsheets

            Can embed invisible Google spreadsheet and look at ``Viewing now''
            on another machine-- how to tell which of these users to serve what
            to? Use lots of different spreadsheets. Assign users to a subset of
            10 spreadsheets, and then chance of overlap pretty low.

            Google fixed by showing user as Anonymous when on a public
            spreadsheet even when logged in, revealing identity only if
            explicitly shared with that user (and the user accepted).
        \end{itemize}
\end{enumerate}
\subsektion{How security bugs contribute to online tracking}
\begin{itemize}
    \item History sniffing and privacy:

        CSS :visited property; how can the web page figure out the color? Check
        in JavaScript
        \begin{itemize}
            \item getComputedStyle()
            \item cache timing: on page, try to download something as embedded;
                if visited before, faster due to caching
            \item server hit: based on if browser downloads image or not
        \end{itemize}
    \item Identity sniffing:
        \begin{itemize}
            \item All social networking sites have groups that users can join
            \item Users typically join multiple groups, some of which are public
            \item Group affiliations act as fingerprint
            \item Predictable group-specific URLs exist
        \end{itemize}
        Look through memebers of groups and see who matches in all the groups
        the user has joined

        Fix as browser: ensure that a site can't see what color a link is by
        keeping track of who (browser's rendering component vs programmatic
        component) is making the query

        Fix as social network: make the URLs not predictable
    \item One-click fraud: Display IP address and approximate location, so user
        assumes the site knows who you are

        What if the website actually has your identity and makes a credible
        threat?
    \item Not a bug:

        Facebook's instant personalization: Facebook tells partner sites who you
        are
\end{itemize}
