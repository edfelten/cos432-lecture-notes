%!TEX root = InfoSec.tex
% Lecture 13: 22 October 2014
\sektion{13}{Web Privacy}
\textit{Note: see piazza for lecture slides}

\subsektion{Third parties}
Third parties (ie. not the site you're visiting), typically invisible, are compiling profiles of your browsing history. There is an average of 64 tracking mechanisms (visible) on a top 50 website, and possibly more invisible ones! 

\textbf{Why tracking?}\\
Behavioral targeting: Send info to ad networks so that user interests are targeted.(Online advertising is a huge and complicated industry)

Trackers include cookies, javascript, webbugs (1px gifs), where a third party domain is getting your information.

\sidenote{
	\textbf{The market for lemons}\\
	George Akerlof, 1971\\
	"Why do people still visit websites that collect too much data?"
	\begin{enumerate}
		\item buyers/users can't tell apart good and bad products
		\item sellers/service providers lose incentive to provide quality (in the case of the internet, privacy)
		\item the bad crowds out the good since bad products are more profitable\\
	\end{enumerate}
}

\textbf{Issues with tracking}\\
\begin{itemize}
	\item intellectual privacy
	\item behavioral targeting
	\item price discrimination
	\item NSA eavesdropping!\\
\end{itemize}

\textbf{Aliasing} \\
If you visit hi5.com with subdomain ad.hi5.com but DNS redirects to ad.yieldmanager.com. Your browser is tricked since this works even if you block 3rd party cookies.

\subsektion{Tagging}
Placing data in your browser

These include etags, cookies, content cache, browsing history, window.name, HTTP STS, etc. With HTTP STS in particular, it can be exploited to tag your browser.

\begin{definition}
	A server can set a flag in a user's browser that says that a certain site can only be accessed securely, through https
\end{definition}

\sidenote{
	\textbf{HTTP STS tagging exploit}\\
	A server can set 1 bit per domain in the sense that the ``must use https'' flag is either on or off. Now, let's say that IDs are 32-bit integers. The server can create 32 subdomains and set tracking bits according to the ID by embedding the request to each subdomain in the page.\\
}

\subsektion{Fingerprinting}
Observing your browser's behavior via things like 
\begin{itemize}
	\item user-agent
	\item HTTP ACCEPT headers
	\item installed fonts
	\item cookies enabled y/n?
	\item screen resolution
\end{itemize}
If you add/hash together all this information, you can likely get a unique fingerprint for your browser! ie. Panopticlick.com

\sidenote{
	\textbf{Anonymous vs. Pseudoanonymous vs. Identity}\\
	Truly anonymous shouldn't be able to track you under a pseudonym in a different session. (How the Internet started)\\

	Pseudononymous can tell when same person comes back but don't know real-life identity. (Internet post-cookies)\\

	Identity can get back to real-life identity.\\
}

\subsektion{More ways for websites to get your identity}
\begin{enumerate}
    \item Third party is sometimes a first party: have first party relationship
        with social networking sites but they're also as widgets on other pages

        Example: Facebook's Like button -- even if you don't click it, Facebook
        knows you were on that page
    \item Leakage of identifiers:

        \begin{tt}
        GET http://ad.doubleclick.net/adj/...\\
        Referer: http://submit.SPORTS.com/...?email=\color{red}{jdoe@email.com}\\
        Cookie: id=\color{red}{35c192bcfe0000b1...}\\
        \end{tt}

        Identity has been compromised now and in the future

    \item Third party buys your identity: free iPod scam passing your email to
        first party site
\end{enumerate}

\sidenote{
    \textbf{Security bugs:}
    \begin{itemize}
    \item http://google.com/profiles/me redirects to \\
        http://google.com/profiles/randomwalker\\

        In firefox, can put the url in a script tag, JavaScript throws error
        which includes the url, giving randomwalker or other identity just
        from visiting this random page\\

        Mozilla's solution: only tell original, not redirected URL

    \item Google spreadsheets: people don't necessarily understand can be public \\

        Can specify all in URL in search to get public spreadsheets\\

        Can embed invisible Google spreadsheet and look at ``Viewing now'' on another machine-- how to tell which of these users to serve what to? Use lots of different spreadsheets. Assign users to a subset of
        10 spreadsheets, and then chance of overlap pretty low. \\

        Google fixed by showing user as Anonymous when on a public spreadsheet even when logged in, revealing identity only if explicitly shared with that user (and the user accepted).\\
    \end{itemize}
}

\subsektion{How security bugs contribute to online tracking}
\textit{This entire section from 2012}
\begin{itemize}
    \item History sniffing and privacy:

        CSS :visited property; how can the web page figure out the color? Check
        in JavaScript
        \begin{itemize}
            \item getComputedStyle()
            \item cache timing: on page, try to download something as embedded;
                if visited before, faster due to caching
            \item server hit: based on if browser downloads image or not
        \end{itemize}
    \item Identity sniffing:
        \begin{itemize}
            \item All social networking sites have groups that users can join
            \item Users typically join multiple groups, some of which are public
            \item Group affiliations act as fingerprint
            \item Predictable group-specific URLs exist
        \end{itemize}
        Look through memebers of groups and see who matches in all the groups
        the user has joined

        Fix as browser: ensure that a site can't see what color a link is by
        keeping track of who (browser's rendering component vs programmatic
        component) is making the query

        Fix as social network: make the URLs not predictable
    \item One-click fraud: Display IP address and approximate location, so user
        assumes the site knows who you are

        What if the website actually has your identity and makes a credible
        threat?
    \item Not a bug:

        Facebook's instant personalization: Facebook tells partner sites who you
        are
\end{itemize}

\subsektion{Defenses}
\begin{enumerate}
	\item \textbf{Referer blocking}

		Two drawbacks: (1) many sites check these for CSFR defense. Blocking all referer headers will break websites.
	\item Third-party cookie blocking

		Relatively little breakage, but doesn't prevent fingerprinting. 

		Safari's cookie blocking policy: It blocks third party cookies unless user is submitting a form, or browser already has cookie from the same party (eg. a facebook ``like'' button still works). 

		Google ad network tried getting around this by using invisible forms
	\item ``Do not track'' option in browsers
	\item HTTP request blocking

		Compile and maintain a list of known trackers based on domain names and regexes. The user installs a browser extension (like adblock plus), which downloads the above list and blocks requests to objects on the list.

		Drawbacks: False positives/negatives; need to trust the list.
\end{enumerate}

