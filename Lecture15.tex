%!TEX root = InfoSec.tex
% Lecture 15: 5 November 2014
\sektion{15}{Backdoors in crypto standards}
\textit{Note: see piazza for lecture slides}

\sidenote{
	USA used navajo as a code during WWII
	\begin{itemize}
		\item couldn't get an accurate phonetic rendering of the code
		\item provided authentication because people can't replicate the tonal language easily\\
	\end{itemize}
}

\textbf{Obvious weaknesses vs. back doors}\\
OBVIOUS WEAKNESS
\begin{itemize}
	\item everyone knows it and can see it
	\item eg. reducing the key size to make brute force attacks easier\\
\end{itemize}

BACK DOOR
\begin{itemize}
	\item presence is not obvious
	\item keyed backdoors vs. unkeyed backdoors

	KEYED BACKDOORS: need a secret master key to access back door

	UNKEYED BACKDOORS: not obvious, but just hoping that someone doesn't notice; like a hidden door in real like
\end{itemize}

\subsektion{Data Encryption Standard (DES)}
(1) introducing a stronger encryption standard would make US communications more secure
(2) but would also allow enemies to find/use this stronger encryption

\textbf{Three organizations working on DES}
IBM, NIST, NSA
DES is a Fiestel-like procedure
\begin{itemize}
	\item had S-boxes with constants (discovered differential crytpanalysis, S-boxes are resistant, but didn't want people to know about differential cryptanalysis)
	\item key length = 56 (NSA wanted to use 48, IBM wanted to use 64, they compromised at 56)
	\item a number of iterations
\end{itemize}
People were concerned about a builtin backdoor. In reality, NSA tried to weaken the algorithm (smaller key size) but installed no backdoor.

\subsektion{DUAL-EC}
\begin{definition}
	Relies on elliptic curves (EC)\\
	Points are solutions to equations of the form 
	$$y^2 \mod p = (x^3 + ax + b) \mod p$$
	You ``add" two EC points to get another EC point. Multiplying point by an int is the same as adding it to itself repeatedly.
\end{definition}

\textbf{How it works}
\begin{itemize}
	\item Pick random, non-secret EC points P, Q
	\item start with secret integer $s_0$
	\item to update and generate new output

		\hspace*{1cm} $s_i = x(s_{i-1} P)$ \hspace{1cm} where x(.) extracts x coord 

		\hspace*{1cm} output $T(x(s_i Q))$ \hspace{1cm} where T(.) truncates, discarding 16 high order bits
\end{itemize}

\textbf{Problem 1}\\
Adversary can create keyed backdoor if they can choose P and Q (see slides)\\
So then naturally the NSA chose P and Q.

\sidenote {
	\textbf{How to generate P and Q?}
	\begin{itemize}
		\item choose random seed
		\item use a one-way algorithm
		\item get P and Q
	\end{itemize}

	But what if adversary chooses the seed?\\
	\hspace*{1cm} It should be okay as long as one-way algorithm is good, and the adversary can't understand the relationship between P and Q
}


\textbf{Problem 2}
Output bits are easily distinguished from random.
NSA argued against fixing this (a vulnerability)

(It's overwhelmingly likely that NSA created a keyed backdoor into this standard by choosing P and Q)

\textbf{What happened}
SSL/TLS are exploitable in practice.
NIST's errors? Not insisting on fixing vulnerabilities, resulting in a loss of trust in NIST.

The end-user was more vulnerable to NSA due to keyed backdoor and more vulnerable to others due to the bias (unkeyed backdoor)\\
\textit{Net effect:} semi-keyed backdoor

\subsektion{Digital Signature standard (DSS)}
There are a lot of known insecure curves, and some curves that are probably secure, but no one can prove it. Let's also say that some adversary can break a fraction $f$ of the believed-good curves.\\

How do we choose a curve when the standard writer might be an adversary?
\begin{itemize}
	\item standard writer chooses a curve (never secure)
	\item choose randomly (secure with Pr = $1-f$)
	\item one-way algorithm with adversary choice; Pr = $(1-f)^k$
\end{itemize}

\subsektion{Backdoor-proof standardization}
\begin{itemize}
	\item Transparency

		discussions on the record, rationale for decisions published
	\item Discretion is a problem! It gives adversary latitude.

		eg. choice of technical approach, choice of mathematical structure, choice of constants
	\item Use competitions, because negotiations in a standards committee is risky.

		participants submit completely proposals. One is chosen by a group and the chosen proposal is adopted as-is or with absolutely clear improvements (this is how AES was gotten)
\end{itemize}

\textit{Shared trust standards benefit everyone :)}
