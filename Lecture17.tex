%!TEX root = InfoSec.tex
% Lecture 17: 12 November 2014
\sektion{17}{Anonymous Communication}
\textit{Guest lecture: Prof. Arvind Narayanan}

Two centuries ago if you wanted a book you went to the library. In a small town, people locally knew what you were viewing. Today, people locally probably don't know what you're viewing, but other people and agencies globally might know.

Crypto for security has clear goals and technical defintitions, but crypto for privacy is fuzzy and often morally ambiguous.

Crypto can help you hide your identity and enables powerful forms of anonymous communication:
\begin{itemize}
    \item Payments
    \item Publishing
\end{itemize}

Is the internet anonymous? {\bf NO}
\begin{itemize}
    \item Best case: pseudonymity
    \item Worst case: identified
    \item Encryption does not hide identities
\end{itemize}
\subsektion{Pseudonymity}
\begin{itemize}
    \item Use of (usually random) psuedonym as ID
    \item Often confused for anonymity
    \item AOL example: AOL released search logs but replaced searcher account names with numbers. Based on the information in the user's searches, the person associated with No. 4417749 was identified in real life.
    \item The problem: one user is always logging into the same acocunt, so search patterns can reveal identity
\end{itemize}

\medskip
{\bf Anonymity via Unobservability}
\begin{itemize}
    \item Hiding the existence of communication {\it or}
    \item Making suspicious traffic look like innocent traffic {\it or}
    \item Piggybacking suspicious traffic on innocent traffic (steganography)
\end{itemize}

\subsektion{Anonymity in Numbers (Tor)}
\begin{itemize}
    \item Lots of people want anonymity
    \item All use the same system/protocol
    \item Adversary can't tell who's who $\Rightarrow$ anonymous within some {\it anonymity set}
    \item How it works: a set of users send their messages through the communication network. The messages are mixed up in the network and sent to the receivers.
    \item Idea: you know which users are using this communication network, but you don't know who sent which message
     \item {\bf Key Idea 1: Trusted Mix} \begin{itemize}
                  \item This is not as anonymous as you may think: the sender must tell the mix the recipient address, and the mix is able to view all the messages
                  \item Problem: aversary can flood the mix with fake senders
              \end{itemize}
     \item {\bf Key Idea 2: Encryption} \begin{itemize}
                  \item Encrypt sender-mix message with public key encryption
                  \item Problem: mix knows both identities
              \end{itemize}
     \item {\bf Key Idea 3: Mix Cascade} \begin{itemize}
                  \item A chain of 3 mixes
                  \item Even if the middle one is compromised, you still can't see the relationship between the input and output of the first and third mixes
                  \item OK as long as one mix is honest
                  \item Problem: but how do we encrypt in this context?
              \end{itemize}
     \item {\bf Key Idea 4: Layered Encryption} \begin{itemize}
                  \item Like an onion, successively encrypt message in reverse order
                  \item Each mix peels off a layer of encryption when sending the message, then forwards
              \end{itemize}
     \item {\bf Key Idea 5: Randomized Routing} \begin{itemize}
                  \item Sender pre-selects random path through network
                  \item All address information is encrypted: when each router decrypts, it finds the address of the next router in the path
                 \item Routers know nothing but previous and next node
                 \item This is secure because an adversary cannot listen to every link on a network
              \end{itemize}
     \item Problem: How can the reciever respond to the sender? \begin{itemize}
                  \item Sender pre-selects return path as well
                  \item Sender sends layered encryption of return path with original message
                  \item Receiver knows only first router in return path
                  \item Each router knows successive router
              \end{itemize}
     \item Problem: High-latency vs. low-latency \begin{itemize}
                  \item So far this works for email
                  \item But it seems too slow for web browsing because you'd need to route each packet. This is computationally costly and more messages being sent leads to more risks. {\it Note: asymmetric encryption is slow.}
              \end{itemize}
     \item {\bf Key Idea 6: Circuit Switching} \begin{itemize}
                  \item Build a circuit for each session, instead of a different route for each packet
                  \item Use above protocol for key exchange, then Alice shares a symmetric key with each router in path
                  \item Use layered symmetric encryption for regular messages within a session, with the same path used for both directions $\Rightarrow$ In reverse, each router {\it adds} a layer of encryption
              \end{itemize}
     \item {\bf Key Idea 7: Tunneling} \begin{itemize}
                  \item Conceptually, Alice $\leftrightarrow$ $R_2$ is tunneled through Alice $\leftrightarrow$ $R_1$
                  \item But it seems too slow for web browsing because you'd need to route each packet. Remember: asymmetric encryption is slow.
              \end{itemize}
     \item Problem: How can Alice trust routers?\begin{itemize}
                  \item Key exchange requires public keys: routers are authenticated
              \end{itemize}
     \item {\bf Key Idea 8: Directory Server (How Tor Works)} \begin{itemize}
                  \item{1.} Alice's Tor client obtains a list of Tor nodes from a directory server
                  \item{2.} Alice's Tor client picks a random path to destination server. All links are encrypted except for the link between the exit node and the receiver
                 \item{3.} If at a later time, the user visits another site, Alice's Tor client selects a second random path. {\it Again the link between the exit node and the receiver is not encrypted.}
              \end{itemize}
     \item {\bf Potential attack: Traffic analysis - timing correlation} \begin{itemize}
                  \item Adversary watches Alice - first router link and last router - Bob link
                  \item The timing patterns are highly correlated, so Tor is vulnerable to a ``global passive adversary"
              \end{itemize}
\end{itemize}

Note: Tor is used for both good and bad, and there's no way to keep only the good uses.

Note: exit nodes can read all unecrypted messages.

Note: Tor does not provide data encryption - the website/server still needs to provide its own encryption mechanism if it wants an encrypted connection.

{\bf Anonymous publishing}: run a webserver without revealing your IP address by picking a random rendevouz point. Both Alice and the server anonymously connect to it.
\begin{itemize}
     \item Silk road is an example
\end{itemize}

